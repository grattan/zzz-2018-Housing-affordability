%!TEX root = ../~/AP-Housing-affordability-2017/Report.tex
\chapter{What can governments do to fix housing affordability?}\label{chap:what-can-governments-do-to-fix-housing-affordability}

So far, this report has analysed what is meant by `housing affordability', and the ways in which housing is -- and is not -- becoming less affordable.
It has explained the causes of these trends, and outlined their implications.
The remainder of the report explores which government policies might improve housing affordability.

Many potential policies might address housing affordability.
But some policies only \emph{appear} to address the problem, rather than actually fixing it.
Some proposals that sound attractive will actually make the problem worse.
And other popular ideas to tackle housing affordability have big budgetary, economic or social costs.

Policy proposals should be judged first on how much they would actually improve housing affordability, and then on their collateral economic, budgetary and social impacts.
Their political feasibility is also relevant.
This chapter provides an overview of policies that are often suggested, and ranks them against these criteria.
Of the policies that will make a significant difference, the Commonwealth should largely focus on policies that will manage demand, and the States should focus on policies that will improve supply.
Subsequent chapters provide the evidence, and detail the evaluation of each proposal.

\section{Criteria for choices}\label{sec:criteria-for-choices}

All the policy changes that would make a difference are politically difficult.
If they were easy, they would have happened already.
Australian governments may find many of the choices unpalatable.
But this report tries to identify all those choices that would make a material difference to housing affordability, and that do not have unacceptable economic, budgetary, or social outcomes.

We believe two criteria are critical in prioritising housing policy reforms:

\begin{enumerate}
\item
  Will the proposal materially improve housing affordability once fully implemented? In other words, will house prices and rents overall be lower (or grow more slowly) than if the proposal were not implemented?
\item
  What are the economic, budgetary, and social impacts of the policy?

  \begin{itemize}
  \item
    Economic impacts: will it have a positive or negative impact on economic activity?
  \item
    Budgetary impacts: how will it affect the budgets of Commonwealth or state governments?
  \item
    Social impacts: how will the proposal affect people and their behaviour?
  \end{itemize}
\end{enumerate}

In addition, some policies are more politically difficult than others.
This shouldn't be decisive.
A good policy is worth pursuing even if it is politically difficult.
With persistent advocacy, public opinion may change, making the politics easier.
But the relative political difficulty does help explain why some policies have been pursued more often than others.

\subsection{Improving housing affordability}\label{subsec:improving-housing-affordability}

House prices and rents matter: they affect home-ownership rates; disposable income for other purposes -- which also affects vulnerability to economic shocks; and inequality, including intergenerational inequality.
House prices and rents can be affected both by reforms that boost housing supply, and by policies that affect housing demand. It is important to look at the \textbf{overall impacts} of policies on house prices and rents.
For example, grants to first home buyers reduce how much individuals pay in the short run, but increase house prices overall in the long run, particularly in areas dominated by first home buyers.
Such policies are unlikely to affect home-ownership, disposable incomes or inequality in the long term, and so score poorly in our assessments.

`Affordability' depends on more than just house prices or rents.
\textbf{What housing is being purchased, and where} also matters.
For instance, building lots of new dwellings far from where people want to live could reduce average house prices relative to incomes, but social welfare would probably not improve much. 
Nevertheless, lower average house prices and rents are a reasonable indication that housing preferences are being satisfied at lower cost.
Affordability also improves if house prices or rents stay the same, but purchase higher quality higher quality or better located housing.

It is also important to look at \textbf{long-term impacts} on house-price and rental growth.
While measures that reduce housing demand are likely to affect prices quickly, measures that increase housing supply are likely to take longer to affect house prices.
The supply of housing depends on the total stock of homes,\footcite[][8]{RBA2015SubmissionHomeOwnershipInquiry} and it takes several years to add materially to this stock.

\subsection{Collateral economic, budgetary and social impacts }\label{subsec:collateral-economic-budgetary-and-social-impacts}

Housing affordability is not the only thing that matters when assessing these policies.
Potential reforms should also be evaluated for their collateral economic, budgetary and social impacts.

Governments face multiple objectives. Although each government sets its own priorities, Australian governments in general have sought to boost material living standards,%
	\footnote{See \textcite[][8]{DaleyMcGannonGinnivan2012-Supportinganalysis} for a discussion of the use of GDP as a measure of economic well-being when prioritising potential economic reforms.}
to maintain balanced budgets over the economic cycle,%
	\footcite[][11--12]{DaleyEtAl-2013-BalancingBudgets}
and to promote better social and environmental outcomes.
Policies to improve housing affordability should be evaluated against these wider goals.

In particular, Australian governments face budget constraints.
With a deficit of \$33~billion a year in 2016-17, the Commonwealth Government has a major budget problem.%
	\footcite[][1]{Treasury-FBO-201617}
Most state government budgets are in surplus, but they face big challenges.%
	\footcite[][15]{DaleyWood2015FiscalChallenges}
Consequently, policies to improve housing affordability must avoid imposing large budgetary costs.
As a result, past measures to provide financial support to first home buyers have largely been wound back, not just because they were ineffective, but because they proved fiscally unsustainable.%
	\footnote{For example see: \textcites{SMH-2015-wrongtimecutfirsthomegrants}{WAtoday-2017-15kfirsthomegrantcut}.}

\subsection{Political difficulty of choices}\label{subsec:political-difficulty-of-choices}

The political difficulty of reform is not an argument for or against any particular reform, but instead helps to illustrate why the housing affordability problem has been so difficult to solve.

Many Australians stand to lose in any attempt to make housing more affordable.
Rising prices are good for some.
As John Howard remarked:

\begin{quote}
I don't get people stopping me in the street and saying, ``John you're outrageous, under your government the value of my house has increased''.%
	\footcite{age-2003-house-prices}
\end{quote}

More recently, the Head of Westpac's Consumer Bank argued that:

\begin{quote}
This whole notion that you want a system where house prices drop is flawed.
It is over \$7 trillion in terms of an asset class.
If that loses value, it would destabilise the economy \ldots{}

This is not about prices going down, this is about ensuring that those who find it difficult to raise a deposit have avenues into getting into home-ownership.%
	\footcite{Smith2017Westpac}
\end{quote}

As these comments reflect, rising house prices affect people differently.
Policies that keep house prices high benefit existing home-owners and housing investors, but hurt those who have not yet bought a house (\Vref{box:who-wins-and-loses}).
Policies that preserve inner-city neighbourhoods are popular among those already living there, because they result in higher house prices, and their lives are relatively undisturbed by the social change that is inevitable when neighbourhood character changes.
As a result, any significant policy change to improve housing affordability is likely to encounter substantial opposition, even if the change is clearly in the public interest.

The political difficulty of a policy cannot be evaluated precisely.
But there are some useful rules of thumb:

\begin{itemize}
    \item People tend to care more about losing something they have already, than potentially gaining an equivalent amount.%
	\footcite{Kahneman2012}
    \item The larger the total loss, the larger the political difficulty.
    \item A small number who will each lose a lot are more likely to organise collectively, and lobby more effectively, than a larger number who will each gain less.%
	\footcite{Olson2009logiccollectiveaction}
    \item Where the absolute amount at stake becomes material for many households, their power at the ballot box can overwhelm more concentrated vested interests
\end{itemize}

These factors make housing a diabolical political issue.
Most voters already own homes.
The value at stake per household is very large -- a home is usually a household's largest single asset.
The `losers' in the current system are those who do not own their own homes -- a less well-resourced minority of the population.

\section{Scope and detail of analysis}\label{sec:scope-and-detail-of-analysis}

The policy proposals we examine cover the main ideas raised in public debate, as well as others where there is strong evidence that they would make a material difference.

Our examination attempts to describe the core of each policy proposal, without trying to analyse every potential variant.
The omitted variations would not usually materially change our evaluation.
But we do typically assess a full-blooded implementation rather than a minor tweak.
For instance, a modest increase in urban infill in the middle-ring suburbs of our largest cities may not concern existing residents much, but it would also only improve housing affordability a little.
Instead we examine what substantial planning reform might look like, and its impacts.

Our estimates of how much prospective policies will affect housing affordability as well as economic, budgetary and social goals should not be treated with spurious precision.
For many of these goals there is no single metric, and their relative importance depends on the weighting of different political values.
Consequently our assessments are often directional, aiming to promote a more informed discussion.

This chapter provides an overview of how policy proposals compare.
Subsequent chapters detail the precise scope of the substantial proposals in more detail.

\section{Summary of key choices}\label{sec:summary-of-key-choices}

Our evaluations of a wide range of policy options are summarised in \Cref{fig:policy-choices-matrix}.

Governments should focus on the policies in the top right of \Cref{fig:policy-choices-matrix}: policies that will make a material difference to affordability without substantially dragging on the economy or the budget.
Almost all of them are measures that would boost the supply of housing.
They include planning changes to facilitate subdivision in the inner and middle rings of our largest cities; boosting density along major transport corridors; and increasing greenfield land supply.

A number of tax reforms to remove distortions in housing investment would have large budgetary and economic benefits, but more modest impacts on housing supply.
These include reducing the 50 per cent capital gains tax discount, limiting negative gearing, including owner-occupied housing in state land taxes, and including more of the value of owner-occupied housing in the Age Pension assets test.
Swapping stamp duties for a broad-based property tax, and improving the provision and efficiency of transport infrastructure would not help the budget, but they would help the economy, and they would improve housing affordability a little.

Almost everything in this category is politically difficult.
Each involves tough trade-offs; each would produce losers as well as winners.
But Australia won't make progress unless it tackles at least some of them.

In contrast, many policies that sound good, but won't help much in practice, live in the top left of \Vref{fig:policy-choices-matrix}.
These include banning self managed super funds from borrowing, and taxes on empty dwellings.
Tighter rules or higher taxes on foreign investors may affect house prices more, but only if set at very high levels, and they may reduce supply overall.

\begin{figure}
\caption{The most popular policies would not improve housing affordability much}\label{fig:popularity-of-policies}
\units{Per cent of respondents that support proposed policy}
\includenextfigure{atlas/Charts-for-housing-affordability-report.pdf}
\source{\textcite{Sheppardetal2017}}
\end{figure}

While those ideas won't do much to make housing more affordable, they won't do much harm either.
Several other ideas, shown in the bottom left of \Vref{fig:policy-choices-matrix},
are less benign: they involve big risks either to the budget or the economy.
For example, allowing seniors to downsize their homes and keep the pension would have big budgetary costs,
 but would make very little difference to affordability because finances are not typically the major motivation for downsizing.
Other proposals, such as stamp duty concessions for first home buyers or allowing them to use their super to fund a deposit would not only cost the budget,
 they would make the affordability problem worse by boosting house prices further.

\begin{figure*}
    \begin{minipage}[t][\textheight]{\textwidth}\vspace{1pt}
\caption{Only some policies will actually improve housing affordability, and these are politically difficult}\label{fig:policy-choices-matrix}
\units{Summary of economic, budgetary, and social impacts}
\includegraphics[page=2]{atlas/policy-choices-chart.pdf}
\noteswithsource{Prospective policies are~evaluated~on whether they would improve access to more-affordable housing for the community overall, assuming no other policy changes.
Assessment of measures that boost households'~purchasing power includes impact on overall house prices.
Our estimates of the economic, budgetary or social impacts should not be treated with spurious precision.
For many of these effects there is no common metric, and their relative importance depends on the weighting of different political values.
Consequently our assessments are generally directional and aim to foster a more informed discussion.}{Grattan analysis}
    \end{minipage}
\end{figure*}

\begin{figure*}
    \begin{minipage}[t][\textheight]{\textwidth}\vspace{1pt}
\caption{Australian governments have largely opted for cosmetic and easy changes that won't improve housing affordability much}\label{fig:policy-choices-matrix-what-govts-have-done}
\units{Summary of economic, budgetary, and social impacts}
\includegraphics[page=3]{atlas/policy-choices-chart.pdf}
\noteswithsource{Where only some state governments have made some progress on a reform, the circle is partially coloured. For example, the NSW Government has made comparatively more progress recently in reforming land use planning rules to boost density along transport corridors, whereas Victoria has done more to boost the supply of greenfield land on the urban fringe. The Victorian Government is also pursuing reforms to tenancy laws. In addition, the ACT Government is replacing stamp duties with a broad based recurrent property tax. The Federal ALP opposition has proposed reforms to the capital gains tax discount, negative gearing and limited-recourse borrowing by self-managed superannuation funds. See also \Cref{fig:policy-choices-matrix}}{See \Cref{fig:policy-choices-matrix}}
    \end{minipage}
\end{figure*}


Measures that would materially reduce demand are mostly Commonwealth responsibilities.
These are detailed in \Chapref{chap:measures-to-manage-demand}.

Measures that would materially boost supply are primarily State responsibilities.
These include planning reforms, making rental more attractive, and improving transport infrastructure so that existing housing has better access to jobs.
These supply-side reforms are detailed in \Chapref{chap:boosting-housing-supply-is-critical-to-make-housing-more-affordable}.

There are many policy changes that will do little to help housing affordability, and these live on the left-hand side of \Cref{fig:policy-choices-matrix}.
Most of the things that governments are actually doing fall into this category (\Cref{fig:policy-choices-matrix-what-govts-have-done}).
These policies are discussed in more detail in \Chapref{chap:proposals-that-wont-help-much}.
Unfortunately, these are also generally the policies that are most popular (\Vref{fig:popularity-of-policies}).

If governments are going to improve housing affordability, they will need to engage more with the public to explain why many commonly suggested policies won't work.
And they will need to lay out the trade-offs involved in policies that will work.

