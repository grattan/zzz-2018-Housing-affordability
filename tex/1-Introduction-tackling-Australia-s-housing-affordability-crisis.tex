%!TEX root = ../Report.tex
\chapter{Introduction: tackling Australia's housing affordability crisis}\label{chap:introduction-addressing-australias-housing-affordability-problem}

\section{People are getting more concerned about housing affordability}\label{sec:public-concern-about-housing-affordability-is-increasing}

Most Australians aspire to the great Australian dream of owning a home.
And most have lived that dream, with ownership rates hovering around 70 per cent since the 1950s. But home-ownership rates are now falling among the young and the poor. Home-ownership increasingly depends on who your parents are, a big turn-around from 35 years ago when all income groups had similar home-ownership rates.
Many struggle for the quality of housing that their parents enjoyed.

Australians of all incomes are spending more of their incomes on housing than they used to. While spending on housing was always going to increase as incomes rose, housing is unnecessarily expensive because land use planning rules restrict the supply of new housing, thereby raising prices. But worsening affordability is hitting those at the bottom the hardest: more low-income Australians are experiencing rental stress and are battling to make ends meet.

\begin{smallbox}{What does `housing affordability' mean}{box:What-does-housing-affordability-mean}

`Housing affordability' is a catch-all term for a grab-bag of public concerns linked to rising house prices.
Some people resent spending more of their pay packet on housing. Some fear that younger Australians will be locked out of home-ownership, and that house prices are widening inequality between and among generations.
Economists are worried that many people can't find housing with good access to jobs.
Others fret about the risks that higher house prices pose to the economy.

In this report we use the term housing affordability because of its familiarity.
We use a broad definition of housing affordability, encompassing a  range of issues, including: 
\begin{itemize}
\item
Households, particularly low-income households, are \textbf{spending more of their income} on housing
\item
\textbf{Rental stress} is increasing for low-income households 
\item
\textbf{Home-ownership rates} are declining, particularly among younger and poorer households
\item
Households are \textbf{taking on more risk} for longer when they purchase a home
\item
Purchasing a first home increasingly \textbf{depends on financial assistance} from family
\item
\textbf{Renting is a poor alternative} for many households
\item
Housing is often \textbf{not being built near where new jobs are} being created
\item
Rising house prices have contributed to \textbf{widening wealth inequality} between and within generations
\end{itemize}

%\textcite{Thomas_Hall_2016_housing_affordability__parl_library} `The term `housing affordability' usually refers to the relationship between expenditure on housing (prices, mortgage payments or rents) and household incomes'
%Rowley and Ong (Housing affordability, housing stress and household wellbeing in Australia 2012 p25) argue for a broad definition of housing affordability, including housing standards and appropriateness, economic participation and social and neighbourhood issues and household composition.
%The rule-of-thumb measure of housing or rental stress: more than 30 per cent of gross income spent on housing by low-income earners

\end{smallbox}

Major cities are increasingly geographically divided, so that young people with less income and education are increasingly concentrated in the fringe suburbs of Sydney and Melbourne.

Cities are not delivering the best mix of housing location and density, given what people would prefer.
And it's getting harder for people to find housing close enough to the job that suits them best.

As a result, public anxiety about housing affordability is growing.
According to one survey, housing affordability has risen in prominence to become the second most important issue people want governments to address (\Vref{fig:public-concern-survey}).


\begin{figure}
\caption{Public concern about housing affordability is rising\label{fig:public-concern-survey}}
\units{Per cent of responses to question, ``What are the three most important issues for government?''}
\includenextfigure{atlas/Charts-for-housing-affordability-report.pdf}
\noteswithsources{Top six concerns in 2017 shown. Some questions change slightly across surveys.}%
{\textcites{AHURI-2017-concerned}{EssentialReport-2017}}
\end{figure}

\section{Fixing housing affordability is politically difficult}\label{sec:fixing-housing-affordability-is-politically-difficult}

Improving housing affordability is a major policy challenge for Australian governments.
Historically, governments have avoided the hard choices, preferring policies that merely appear to address the problem.

The politics are difficult because many more people are home-owners than aspiring home-owners.
In a typical year, roughly 100,000 people in Australia will become home-owners for the first time.%
\footnote{Estimated from first home buyer finance commitments  (\textcite{ABSHousingFinanceAustraliaAugust2017}).}
They would benefit from any policy that leads to lower house prices.
But at the same time, 5.4~million households already own at least one property.\footnote{In 2016 there were 9,326,000 private dwellings, of which 8,286,000 were occupied on Census night.} 
Policies that might result in lower house prices are not in their interests.
Former prime minister John Howard famously remarked that people did not complain to him about the price of their house going up.%
	\footcite{age-2003-house-prices}
But older generations do worry about whether their children and grandchildren will be able to buy a house.
And rising house prices don't necessarily help home-owners planning to buy a more expensive home in the future.

Addressing housing affordability will inevitably lead to house prices and rents growing more slowly than otherwise, or even falling.
But gradual change is desirable: a sharp fall in house prices would not only be politically very unpopular, it would probably cause economic upheaval and do more harm than good.
Reform of housing policy settings is therefore a sensitive and long-term project.


\section{What this report does not do}\label{sec:what-this-report-does-not-do}

This report does not examine in detail how governments can best help \textbf{very low income earners meet their housing needs}.
As this report shows, there is a powerful case for additional support to help those worst off to cope with rising housing costs: low-income households are now much less likely to own their own home, and their rents are increasing faster relative to incomes.
As outlined briefly in \Vref{sec:helping-the-bottom}, this may require increasing the social housing stock, which represents a small, and declining, share of the overall housing stock.
But the public subsidies required to make a real difference would be very large.
Alternatively, support may also come in the form of increased financial assistance for low-income earners who are experiencing financial stress in the private rental market.

While very important, a more comprehensive review of the challenges of rising housing costs for low-income earners is beyond the scope of this report.
Others have worked extensively on these important issues,%
	\footnote{See \textcites{Yates2016why}{Pawson-et-al-2015-Addressing}{Milligan-2016-profiling}{SenateEconomicsRefAffordableHousing2015}{Rowley-etal-2017-Govt-led-innovations-affordable-housing-delivery} for policy suggestions.}
which typically require specific policy interventions in addition to those described in the rest of this report applying to the remaining 95 per cent of the housing market.

\section{How the rest of this report is organised}\label{sec:how-the-rest-of-this-report-is-organised}

The following chapters describe in more detail what's happened to housing affordability, what problems this has created, and what governments should do about it.

\Chapref{chap:housing-is-less-affordable} shows how \textbf{housing has become less affordable}, tracking trends in house prices and rents over the past two decades.

\Chapref{chap:the-causes-of-higher-housing-costs} analyses \textbf{why house prices have risen so much} in recent decades, especially compared to incomes, and why rents haven't followed suit.

\Chapref{chap:worsening-housing-affordability-has-serious-consequences} considers the \textbf{effects of worsening housing affordability on peoples' lives}.
Rapid growth in house prices has lowered home-ownership rates among younger and poorer households, contributed to widening wealth inequality, and left the economy more vulnerable to economic shocks.
This chapter also describes why renting is often seen as a poor substitute to owning a home.

\Chapref{chap:what-can-governments-do-to-fix-housing-affordability} outlines our \textbf{framework for how governments should prioritise reforms to improve housing affordability}.
Governments should consider whether a policy materially improves housing affordability, and also the social, economic and budgetary impacts of the policy.

\Chapref{chap:measures-to-manage-demand} outlines \textbf{what governments should do to manage the growing demand for housing.}
This includes potential tax reforms, new macro-prudential rules, and changes to migration rules.

\Chapref{chap:boosting-housing-supply-is-critical-to-make-housing-more-affordable} shows that \textbf{boosting housing supply in our major cities is key to improving affordability}.
State governments control the biggest lever: reforming planning rules to allow more housing to built in inner and middle suburbs of our major cities.

\Chapref{chap:proposals-that-wont-help-much} discusses \textbf{recent government policies and proposals that appear to improve housing affordability, but won't do much} in practice, such as extra tax breaks for first home buyers, and incentives to live in regions.

\Chapref{chap:Conclusion} identifies \textbf{what it would take} for Australian governments to turn around decades of policy failure and build momentum to make housing more affordable.

