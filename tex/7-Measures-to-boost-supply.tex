%!TEX root = ../Report.tex
\chapter{Measures to boost supply}\label{chap:boosting-housing-supply-is-critical-to-make-housing-more-affordable}

The previous chapter showed that the Commonwealth and state governments can improve housing affordability a little by reducing demand -- largely by reforming taxes and concessions that inflate how much people are prepared to pay for housing.
But governments could improve affordability much more over the medium term by increasing supply.
This is primarily a problem for state governments: they set the overall framework for land and housing supply, and they govern the local councils that assess most development applications.

As noted in \Vref{sec:rising-house-prices-are-primarily-due-to-rising-land-values-not-construction-costs}, housing prices have increased primarily because of increases in the price of land, not increases in the cost of buildings.%
	\footcite{Knoll-et-al-2017-no-price-like-home}
Higher land prices mainly reflect restrictions on supplying more dwellings: much urban infill is limited by planning restrictions; and greenfield development at the urban fringe is often limited by slow release of land, planning approval delays, and uneconomic developer charges, particularly in Sydney.
Land use regulation is also seen as the major contributor to higher dwelling prices in large cities in many other developed economies.%
	\footnote{In the US (see \textcite{GlaeserGyourko2003BuildRestriction}),
     England (see \textcite{Hilber-Vermeulen-2015-Supply-constraints-effect-on-English-house-prices}),
     and New Zealand (see \textcite{Lees2017LandRegulation}).}

In Australia, development restrictions have been most stringent in the established suburbs of major cities with better access to places where more jobs are being created.
Unsurprisingly, land values have risen fastest in these inner-city areas.

Limits on additional dwellings caused more problems because migration has rapidly increased demand for new dwellings. As a result, new housing construction fell well behind population growth and demand for additional housing. Although construction rates have increased in the past few years, they remain below what is needed given current population growth.

The failure to permit enough development reflects the fraught politics of NIMBYism.

The problems of housing affordability outlined in this report will mostly get worse unless state governments move on supply. Their highest priority should be building the public case for increased density, and then changing planning and approval processes accordingly.
State governments should also increase the supply of greenfield land and ensure that excessive developer charges do not make development uneconomic -- although they should try to capture more of the windfall profit when land is rezoned or permits are granted.

State governments can also make good-quality housing more affordable by improving rental conditions.
They should reform residential tenancy acts to provide renters with more secure tenure.
They should eliminate tax-free thresholds and flatten rates of land tax to make large-scale institutional investment in housing economic -- which would lead to more secure tenure.
And they should put a greater priority on transport infrastructure projects that service the biggest increases in actual population growth.

Such policy changes would make a real difference -- but they would require step changes in political will, real public engagement, and a sustained approach for many years.

While these supply reforms are largely state government responsibilities, the Commonwealth still has an important role.
Because Australia's housing markets are interconnected, no state government can solve the housing affordability problem alone.
If a state government substantially boosts housing supply in one state capital, any improvement in affordability will be dispersed across Australia.
The Commonwealth can help solve this coordination problem.
And it has an interest in doing so: it will ultimately reap much of the benefit of increased economic growth encouraged by higher tax revenues that flow from better housing policies.
Consequently the Commonwealth should provide incentive payments to the states to boost housing supply and reform state property taxes.

Because low-income households are now much less likely to own their own home, and their rents are increasing faster relative to incomes, there is a powerful case for more public support to help them with rising housing costs.
But the public subsidies required to make a real difference to current arrangements would be large.
State governments could also adopt `inclusionary zoning' policies that compel new developments to include a proportion of new social housing.
But this would be a large-scale change to Australia's development market, which could have big unintended consequences.
Consequently new policies to promote housing with sub-market rents need to be designed carefully.
Eligibility and allocation criteria for public and community housing also need reform.

\section{State governments need to increase housing supply in our major cities}\label{sec:state-governments-need-to-increase-housing-supply-in-our-major-cities}

The need for more housing supply, especially in Sydney and Melbourne, has been the focus of analysis and government policy for some time.
As outlined in \Vref{subsec:housing-construction-did-not-keep-pace-with-increased-demand-and-prices}, on most estimates, dwellings fell well behind population growth for the decade from 2005-14.
Construction has only started to get close to matching population growth in the past couple of years; the backlog of a decade of under-supply remains.
If projected population growth rates are right, then future rates of construction will need to be even higher than current elevated levels.

And the existing housing stock is often a poor match for peoples' preferences when they trade off house size, style and location.
There appears to be substantial unmet demand for medium-density housing, particularly in the middle ring of Melbourne and Sydney (\Vref{sec:supply-has-not-matched-consumer-preferences-well}).
Such housing is also particularly important for the economy, with high-value jobs growing rapidly and tending to cluster towards the centre of our major cities.

\subsection{More housing supply is key to improving affordability }\label{subsec:boosting-housing-supply-in-our-major-cities-is-key-to-improving-affordability}

Boosting housing supply would substantially reduce house prices in the medium-term.
Reviews by the Productivity Commission and several others have identified boosting housing supply as a key to improving affordability.%
	\footcites{ProductivityCommission2004FirstHomeOwnership}{RBA2014SubmissionAffordableHousingInquiry}{SenateEconomicsRefAffordableHousing2015}{Stevens-2017-Report-to-NSW-Premier-Housing-affordaibility}[][15]{IMF2018_ArticleIV}
The effect is potentially large. Adding 1 per cent to the housing stock leads to dwelling prices between 1 and 3.5 per cent lower than otherwise.%
	\footnote{The lower bound estimate of 1 per cent is based on demand elasticity of -0.6 (\textcites{Albouy-2016-housing-demand}{barkerinterim2003}) and supply elasticity of 0.3 (\Vref{subsec:housing-construction-did-not-keep-pace-with-increased-demand-and-prices}).
	The upper bound estimate of 3.5 is from \textcite{Abelsonetal2005}.}
Relatively small changes in the stock of dwellings can have big impacts because people looking for housing have few alternatives: ultimately their next best choice is often to live in a larger household.

Of course, relaxing land use planning restrictions to build more housing in attractive inner and middle ring suburbs will increase the value of that land per square metre.%
    \footnote{For example, \textcite[][11]{KulishRichardsGillitzer2011} find that residential building height restrictions result in lower land prices closer to the CBD where the height restriction is binding.
    See \textcite{Brueckner2007} for a theoretical overview of the impact of land usage policies on land prices.}
However, the price of land needed \textit{for each dwelling} should be lower than otherwise. 

Some commentators and academics argue that boosting housing supply won't make much difference to housing affordability.%
	\footnote{\eg~\textcites{RowleyGurranPhibbs-2017-worldleaderhomebuilding}{Pawson-2017-theConvo-To-do-list-for-NSW-Premier}.}
But in the medium run, housing \textit{is} like bananas:%
    \footcite{Gurran_Phibbs_2014_bananas_conversation}
more supply leads to lower prices than otherwise (\Vref{box:why-supply-matters}).

Of course, boosting housing supply will only improve affordability slowly. Even at current record rates, new housing construction increases the stock of dwellings by only about 2 per cent each year. According to available estimates, adding an extra 50,000 dwellings to Australia's housing stock -- an increase of about 25 per cent on current levels of construction nationally, or roughly 0.5 per cent of the national housing stock -- would lead to national house prices being only 0.5 to 2 per cent lower than otherwise.%
	\footnote{Based on \textcites{Abelson-2016-Housing-costs-policies}{Albouy-2016-housing-demand}{barkerinterim2003}; see also \Vref{subsec:housing-construction-did-not-keep-pace-with-increased-demand-and-prices}.}
This is much lower than the typical annual price rise of the past few years.

But these estimates also imply that a \emph{sustained} increase in housing supply would have a big impact on house prices.
For example, if an extra 50,000 homes were built each year for the next decade, national house prices could be between 5 and 20 per cent lower than they would be otherwise. Over a longer period prices would be even lower. 
In the past, additional supply over the long run has successfully limited price growth, even when the population grew rapidly.%
    \footnote{Between between 1947 and 1961, Australia’s population increased by 41 per cent -- while the housing stock increased by 50 per cent -- and house prices were broadly stable (\textcite{Eslake2013}).
    Similarly, \textcite[][31--32]{Glaeser-2013-Natio-of-gamblers} notes that rapidly expanding housing demand in the U.S. immediately after World War II was met not by rising house prices, but by a sustained increase in home-building, especially on the urban fringe and in Sunbelt states.}

The increase in housing construction over the past few years, particularly the construction of apartments in Brisbane, Sydney, and Melbourne (see \Vref{fig:CKC-region-timeseries}), has ultimately kept prices and rents lower than otherwise.%
	\footnote{A large number of apartments remain in the pipeline for completion over the next few years: \textcite[][Graph~3.14]{RBAStatementonMonetaryPolicy_Nov17}.}
The effects are most noticeable in the Brisbane apartment market, with well-publicised falls in inner-city apartment prices and low growth or declining rents.%
    \footnote{See \Vref{box:brisbane-council-2014} and \textcite{hamilton_smith_2018_brisbane_apartments}.}
The apartment construction boom in Sydney's inner and middle suburbs that took off in 2013-14 is also beginning to affect prices noticeably.
Sydney house and unit prices have been falling in recent months, and there are reports that rents are declining in suburbs where lots of new apartment developments have been completed.%
	\footcite{Devine_2018_new_sydney_rents}
Similarly, in Melbourne, price growth for apartments (and particularly inner-city apartments), has been slower than price growth for houses.%
	\footcite{ABS-2017-Residential}
Of course, many factors determine price growth in any particular period.
For example, APRA's 2017 crackdown on interest-only loans has also contributed to the slowdown in prices (\Vref{sec:macro-prudential-rules-should-be-used-to-manage-financial-sector-risks-where-required-but-not-to-reduce-house-prices}).

\begin{bigbox}{Scepticism about the links between planning, housing supply, and prices}{box:why-supply-matters}

A number of academics are sceptical that planning is limiting housing supply, or that limited housing supply is driving house prices up. Some argue that \textbf{planning rules have not been the primary limit to housing supply}.%
    \footnote{\eg~\textcite{OngEtAl-AHURI-2017-Housing-supply-responsiveness} conclude that `restrictiveness of planning measures is unlikely to be the key factor in influencing housing supply'. See also \textcite{Gurran_Phibbs_2016_boulevard}.}
But these conclusions are contradicted by a large volume of evidence (\Cref{box:International-supply-literature}). 
And the fact that development is not delivering residents the type of housing -- by location and density -- that they say they prefer (\Vref{sec:additional-supply-has-not-matched-consumer-preferences}) suggests that planning restrictions \textit{are} getting in the way of supply and demand. 

Another argument is that in practice \textbf{housing supply does not respond to demand}. This argument claims that most new buildings have been more expensive than the existing stock.
But the claim was built on a flawed analysis of the data: new housing has \emph{not} differed much from the existing stock in price, and in any case, when residents move into new higher priced homes, this frees up housing for those with lower incomes (\Cref{box:Ong-box}).    
 
Others argue that \textbf{housing supply is not the major cause of housing affordability problems}.
They point out that many more dwellings were constructed in Sydney and Melbourne from 2013, and yet housing prices continued to rise rapidly.%
    \footnote{\textcite{RowleyGurranPhibbs-2017-worldleaderhomebuilding}. See also \textcite{Pawson-2017-theConvo-To-do-list-for-NSW-Premier}.}
Of course, limited supply is not the \textit{only} influence on housing prices.
Most importantly, the RBA's official cash rate halved between~2011 and 2016.
But in any case, a short run increase in the \textit{flow} of new housing will not have much effect on housing prices, which depend on the imbalance between the \textit{stock} of supply and demand.
In 2013, there was a large overhang of latent demand because population growth had outstripped new housing construction for most of the previous decade (\Vref{subsec:housing-construction-did-not-keep-pace-with-increased-demand-and-prices}).

Indeed, one would expect that \textit{in the short run}, more supply is generally associated with rising prices.
If prices are higher, then typically housing supply increases, because developers find it easier to access finance for new projects.
But that doesn't mean supply \emph{causes} higher prices. If dwelling construction had not increased in Sydney and Melbourne from 2013, housing prices would have risen even further. 

Several other features of Australian housing reflect how supply, demand, and prices are linked, as in most markets.
\begin{itemize}
\item
Most of the increase in the value of housing is driven by higher land prices not construction costs -- which suggests that additional dwellings are scarce (\Vref{sec:rising-house-prices-are-primarily-due-to-rising-land-values-not-construction-costs}).
\item
Land values rose faster in inner suburbs (\Vref{fig:house-prices-within-cities}) where planning controls are tighter (\Vref{sec:the-major-problem-is-planning-regulations}).
\item
Land values rose faster on the fringes of Sydney where less land was released than in Melbourne (\Vref{fig:greenfield-land-three-panel}).
\item
Younger Australians are living with their parents for longer, and fewer 20-34 year olds start their own household in Sydney and Melbourne where house prices are highest (\Vref{subsec:undersupply-led-to-larger-households}; \Vref{subsec:rising-house-prices-increase-the-risks-that-younger-generations-will-be-worse-off-than-their-parents}).
\end{itemize}

The better view is that \textit{in the long run} planning has limited housing supply and increased prices.
Reviews by the Productivity Commission and several others have come to a similar conclusion.%
    \footcites{ProductivityCommission2004FirstHomeOwnership}{RBA2014SubmissionAffordableHousingInquiry}{SenateEconomicsRefAffordableHousing2015}{Stevens-2017-Report-to-NSW-Premier-Housing-affordaibility}


\end{bigbox}

\subsection{Changes to enable more supply are politically very difficult}\label{subsec:changes-to-enable-more-supply-are-politically-very-difficult}

Planning regulations have not changed much, despite the pressure of increasing population, because the politics of planning are poisonous.
Most people in the established middle suburbs already own their house.
Most of them don't like new developments in their neighbourhoods -- the NIMBY syndrome.%
	\footcites{Visentin-2017-SMH-Housing-afford-lessons-from-Vancouver}{Robertson-2017-SMH-Sydenham-Bankstown-redevelop-study-developers-could-add-25k-homes}
And so most people in Sydney believe that additional population should be housed primarily outside the existing Sydney boundaries.%
	\footnote{The Productivity Commission reported a survey in which more than half of Sydney respondents said they would not like an increased population in their neighbourhood: \textcite[][28]{PC2011PerformanceBenchmark}.
	Two-thirds of respondents to a recent survey agreed that `Sydney is full and we should push development outside metro Sydney' (\textcite{Nicholls-SMH-2017-survey-city-is-full}).}

The structure of government doesn't make the politics of increasing density any easier.
The voting bases of councils, the basis on which they collect rates, and the blurring of responsibilities between the Commonwealth and the states all reduce the political incentives for any level of government to do better.

The benefits of population growth accrue to society as a whole, whereas decisions about development approvals largely sit with local councils. 
\emph{Existing} residents usually prefer their suburb to stay the same.
Restricting development effectively increases the scarcity value of their property.
And they worry that increased population will reduce the value to each of them of the current publicly provided infrastructure in their area such as roads and other amenities: existing residents are typically concerned that there will be more traffic congestion, more crowding on public transport, more noise and less `street appeal' (\Vref{fig:reasons-dont-want-pop-growth}).

\begin{figure}
\caption{There are many reasons people don't want the population of their neighbourhood to increase}\label{fig:reasons-dont-want-pop-growth}
\units{Per cent of respondents, 2011}
\includenextfigure{atlas/Charts-for-housing-affordability-report.pdf}
\notewithsource{Respondents could choose multiple reasons and so totals do not sum to 100}%
{\textcite[][28]{PC2011PerformanceBenchmark}}
\end{figure}

Meanwhile, \emph{prospective} residents who don't already live in middle-ring suburbs can't vote in council elections, and their interests are largely unrepresented.

The regulation of council revenues by state governments can give councils additional reasons to oppose development.
Many councils are `rate-capped': state governments limit how much they can increase rates per resident per year.%
	\footnote{See \eg~\textit{Local Government Act 1989} (Vic), ss.\ 185A-185G.}
Consequently, new developments effectively increase the rates for existing residents.
When land is subdivided, the rates for each subdivided property are usually smaller than for the original property, even if collectively they are larger.
But each new property usually consumes council services at a similar rate per person.
To stay within overall average rate caps per property, councils impose lower rates on the subdivided properties, but higher rates on existing residents of larger properties.%
	\footnote{Calculating rates on the capital-improved value rather than the value of the land can at least reduce this effect.
	While in theory this might discourage development, the tax rate is typically so low that it is little disincentive to development in practice: \textcite[][4]{DaleyCoates-2015-Property-taxes}.
	Consequently, \textcite{IPART-2016-Review-of-local-govt-rating-sys} recommended that the NSW Government give councils the option to use the capital-improved value of a site as an alternative to unimproved value as the basis for setting the variable amounts in council rates.}
While councils subject to rate-capping can apply for Special Rate Variations (SRVs), they are reluctant to do so even when clearly necessary because exceeding the cap excites comment, and so is considered politically risky.%
\footcites[][42]{Sansom-NSW-Ind-Local-Govt-Review-2013}{IPART-2012-submission-to-Local-Govt-Review}

The division of responsibility between different levels of government also discourages difficult decisions.
No single level of government owns the challenge of managing population growth in our biggest cities.
And so no government is responsible for the serious consequences of failing to plan for growing populations.
Instead, more housing will be built on the urban fringe where there are no existing residents to object, but it will be far from jobs and existing infrastructure.
And house prices will keep rising.

\subsection{State governments should communicate the benefits of increased housing density in our largest cities }\label{subsec:state-governments-need-to-communicate-the-benefits-of-increased-housing-density-in-our-largest-cities}

The politics will only change if more people understand the trade-offs in failing to develop more housing.
Public engagement is vital.
It provides the framework for residents to think about choices facing their cities and neighbourhoods.
Residents usually engage in the planning process only to respond to specific development applications rather than to think through proposals on how the whole neighbourhood should change over time.
The few examples of successful reform suggest that the public will only accept population growth in their neighbourhoods if residents are actively involved in a long-term discussion about the future of their city \emph{and} their neighbourhood.%
	\footcites{KellyHarrisonHunterEtAl2013}{Kelly-2010-Cities-who-decides}

State governments need to clearly and repeatedly lay out the trade-offs in development.
They need to spell out how more medium-density dwellings in established areas are exactly the kind of dwellings that current residents would like their children to buy.
And they should explain that this is also the kind of housing that existing residents will probably want to downsize into in a few years' time.%
	\footnote{\textcite{Daley-2017-AAA-Housing-for-older-Australians-COTA-prez}; \Vref{subsec:tax-settings-discourage-people-from-downsizing-increasing-demand-for-well-located-houses}.}
They need to articulate -- and then deliver -- the additional services that will be available if there is greater population in an existing area: better services such as improved infrastructure, more shops, more community facilities, and communal green space.%
	\footcite{Sweet-2010-Why-medium-density-is-health-issue}

Governments need to reassure existing residents that adverse impacts will be limited.
Often this requires clearer and \emph{less} flexible rules around what is acceptable development and what is not.
And they need to ensure that new medium-density housing is well-designed and well-constructed.

\subsection{States should reform planning rules to make it easier to develop medium-density housing in middle-ring suburbs}\label{subsec:states-should-reform-planning-rules-to-make-subdivisions-in-middle-ring-suburbs-easier-and-engage-communities-in-suburban-development}

State and local governments need to change planning laws and practice to make it easier to subdivide and increase housing supply in middle-ring suburbs.

Current rules and community opposition make it very difficult to subdivide and create extra residences in the inner and middle rings of the capital cities (\Vref{sec:the-major-problem-is-planning-regulations}).

Grattan Institute's 2011 report, \emph{Getting the housing we want}, recommended a new Small Redevelopment Housing Code that would protect neighbours, reduce planning uncertainty, and improve the quality of new developments.%
	\footcite[][26--27]{KellyBreadonReichl2011}
The Code would include the things that worry neighbours most: such as privacy, height and overshadowing of their outdoor areas, and the appearance of new developments from the street.
It would cover all developments that provide two to ten new dwellings, depending on the lot size, and are one or two storeys high.
Builders who do not comply need to be forced to do so -- or have their building demolished at the expense of those who assessed it as compliant.%
	\footnote{In addition, some kind of bond deposit or pooled insurance scheme might be required so that this `make good' threat is credible, \textcite[][230]{CommissionMigrantIntake2016}.}
And it would apply to all residential areas, unless there are heritage or environmental restrictions.

The NSW, Victorian, and Queensland governments are making the right noises about boosting new housing supply, but are not yet doing enough.
Their recent housing affordability packages contain some promising initiatives,%
    \footnote{See \Vref{subsec:some-planning-reforms}.}
but also avoid making some of the tough decisions that would boost housing supply in the middle-ring suburbs where many people want to live.

\begin{bigbox}{Brisbane City Council's 2014 land use planning rules changes}{box:brisbane-council-2014} % TODO: footnotes across both columsn

The Brisbane City Council's \textit{City Plan 2014} substantially changed land use in Brisbane. The plan began with extensive community consultation. In 2005 and 2006, over 60,000 residents contributed to the CityShape plan which projected how Brisbane should grow over the following two decades.%
    \textsuperscript{a}
	% \footcites{BCC2014_cityshape}{BCC2014updatetodraft}
The CityShape plan was then used during the five years of community consultation prior to the commencement of the Brisbane City Plan.
During the consultation over the CityShape plan, residents could attend information sessions and `Talk to a Planner' sessions and make formal submissions.%
	\textsuperscript{b}
The Council received just over 2,700 submissions on the draft plan, and made some changes before implementing the final plan on 30 June 2014.

The plan aimed to align the development assessment process with the expectations created by the plan and to increase infill development near the CBD and along transport corridors.%
    \textsuperscript{c}
    % \footcite{BCC2014updatetodraft}
The plan made some apartment developments `code assessable', meaning that a development application is determined simply by reference to the applicable code. The public are only notified of a proposed development if it exceeds the height limits of the code. This shortens the approval process for most developments.%
    \textsuperscript{d}
    % \footnote{\textcites{Shoory2016Apartment}{BCC2017BrisbaneCityPlan}{BCC2017_multipledwellings}.}
Apartment developments from 10 to 20 storeys high in the high-density residential zone, covering the CBD and parts of inner-city suburbs such as Spring Hill, Kangaroo Point and West End, became code assessable.
Apartment buildings up to five storeys in the medium-density residential zone also became code assessable.

\vfill
\rule{0.2\columnwidth}{0.4pt}\linebreak
\begin{tabularx}{\columnwidth}{@{}>{\centering\footnotesize}p{1.5em}@{}>{\footnotesize\arraybackslash}X}
        a.~\null &
            \textcites{BCC2014_cityshape}{BCC2014updatetodraft} \\
        b.~\null &
            {{\textcite{BCC2014_community_consultation} and information obtained from consultation with stakeholders.}} \\
        c.~\null &
            \textcite{BCC2014updatetodraft} \\
        d.~\null &
            {\textcites{Shoory2016Apartment}{BCC2017BrisbaneCityPlan}{BCC2017_multipledwellings}.} \\

\end{tabularx}

\eject

These reforms, in combination with strong demand for housing, spurred a wave of construction, particularly of inner city apartments.%
    \textsuperscript{e}
	% \footcite[][28]{PC-2017-shifting-dial-potential-of-land}
In the two years from July 2014 when the plan took effect, there were over 30,000 approvals for non-detached dwellings in the Brisbane City Council area, up from just under 15,000 in the preceding two years.%
    \textsuperscript{f}
% \footcites{ABS-2017-Building-approvals-Sep-2015}{Shoory2016Apartment}
Apartment completions in Brisbane jumped from around 4,000 in 2014 and 2015 to around 11,000 in 2016 and 2017.
Most of the development was in the CBD and immediate surrounds, and over 10 stories (\Cref{fig:CKC-region-timeseries} and \Vref{fig:CKC-storeys-timeseries}).

Prices for apartments in Brisbane have grown more slowly in recent years compared to apartments in other capital cities, and have also grown more slowly than houses in Brisbane, in part due to the construction boom.%
	\textsuperscript{g}%\footcite{ABS-2017-Residential}
Since 2016, apartment prices in Brisbane have fallen and rents have grown more slowly.%
    % \footnote{There are numerous reports of significant discounting by vendors and landlords, \eg~\textcite{hamilton_smith_2018_brisbane_apartments}.}
    \textsuperscript{h}
The slower price growth of Brisbane apartments is due in part to the strong increase in the supply of new housing as developers were more easily able to build housing to meet demand.

The Brisbane City Plan 2014 is a good example of how community consultation about how to house a growing population can lead to codified planning rules to encourage infill development. And this led to more housing supply and cheaper housing.

\vfill
\rule{0.2\columnwidth}{0.4pt}\linebreak
\begin{tabularx}{\columnwidth}{@{}>{\centering\footnotesize}p{1.5em}@{}>{\footnotesize\arraybackslash}X}
        e.~\null &
            \textcite[][28]{PC-2017-shifting-dial-potential-of-land} \\
        f.~\null &
            \textcites{ABS-2017-Building-approvals-Sep-2015}{Shoory2016Apartment} \\
        g.~\null &
            {\textcite{ABS-2017-Residential}} \\
        h.~\null &
            {There are numerous reports of significant discounting by vendors and landlords, \eg~\textcite{hamilton_smith_2018_brisbane_apartments}.} \\
\end{tabularx}\par
\end{bigbox}


\subsection{States should set housing targets and make sure councils meet them}\label{subsec:state-governments-need-to-set-and-make-councils-meet-future-housing-targets}

Whatever the formal requirements of the planning scheme, for the foreseeable future local councils are likely to retain the power to decide which zone should be applied to much of the land in their municipality, and substantial discretion to approve (or reject) development applications.

Governments have often tried to tip the balance in favour of more housing by setting housing targets for individual councils.\footnote{The Victorian Government's \emph{Plan Melbourne} does not have housing supply targets for local councils, but these will be developed in coming years as part of the five-year implementation plan (Action numbers 1 and 19 of \textcite{VicGov2017PlanMelb}).}
The Greater Sydney Commission process is the latest such effort.
But experience shows that such targets do not work unless:

\begin{itemize}
\item
The overall housing targets for each council, and the translation into plans for particular areas, are realistic;
\item
These targets and plans are the product of a process that engages the community in understanding the rationale for increased housing; and
\item
There are real consequences for councils that do not meet their targets.
\end{itemize}

Housing targets for each council need to be linked to overall plans for the growth of the city as a whole.
Each council then needs to identify how its target will translate into additional housing for each particular area within its jurisdiction.
These plans can be linked to state government commitments to improve local infrastructure.
Obviously the overall targets will not be delivered unless the additions planned for each nominated location are realistic given what land use planning rules are likely to deliver.%
	\footcites[][10]{BuxtonEtAl2015}{Property-Council-2015-Missing-the-mark}

Both the target for each council, and the translation into plans for particular areas need to involve residents so that there is a substantial body -- if not a majority -- of opinion within the area that understands the underlying rationale for change, and the trade-offs involved.
Otherwise councils are likely to avoid meeting the target.
As discussed in \Cref{subsec:changes-to-enable-more-supply-are-politically-very-difficult}, the politics of planning are very difficult.
As \Vref{subsec:state-governments-need-to-communicate-the-benefits-of-increased-housing-density-in-our-largest-cities} shows, the politics are only tractable if governments engage in an extended process so that the entire community understands the trade-offs of development.

Delegating responsibility to an intermediate authority such as the Greater Sydney Commission to set targets may defer political opposition, but it is unlikely to result in sustainable policy once an outcome emerges that lacks popular support.

Given the difficult politics of planning, some councils are likely to try to water down commitments or delay changing strategic plans, even if there has been considerable community consultation. Governments should impose credible enforcement mechanisms to prevent back-sliding and to assure each council that other councils are pulling their weight.
In the past, processes for setting housing targets for population growth in each local council area have often failed because they lacked any credible enforcement mechanism such as incentive payments or penalties for non-compliance that can realistically be used by the state government in the face of significant public disquiet.

State governments need to carry bigger `sticks' to ensure councils meet the housing targets included within state strategic plans.
These might include creating powers for the state government to take over authority for a larger share of development approvals if councils fail to back appropriate development.%
	\footnote{For example, see \textcite{Daley-etal-2017-Submission-NSW-housing-supply-inquiry}.}
State governments could also offer `carrots', such as bonus payments for councils that meet or exceed housing targets.%
	\footnote{For example, the NSW Government is providing up to \$2.5~million for each priority council to update their Local Environment Plan (LEP), and incentive payments to other councils that volunteer to update their LEPs (\textcite{NSWGovFirstHome2017}).}
Obviously these bonuses need to be large enough to outweigh the political costs to councils of pro-development decisions.%
	\footnote{Incentives from the NSW Government of \$2.5~million per council seem pretty small relative to the political cost, and much higher bonuses may be appropriate considering the value of land rezoning and the productivity benefits of more housing in well-located areas.}

Many previous attempts to set housing targets for Sydney or Melbourne have failed because they did not follow these steps.%
	\footcites[][Table~2]{Property-Council-2015-Missing-the-mark}[][9]{UDIA-NSW-2017-Making-housing-more-affordable}[][13]{Property-Council-2017-submission-NSW-inquiry}[][46]{GSC-2017-draftplan}

The current reforms may do a little better, although some signs are not promising.
The NSW Government has empowered the Greater Sydney Commission to set housing targets for the five districts and local councils.%
	\footcite{NSWGovFirstHome2017}
These targets are based on the projections in the Commission's \emph{Draft Greater Sydney Region Plan}.%
	\footnote{The Draft Greater Sydney Region Plan is an update to the 2014 plan, `A Plan for Growing Sydney', and the 2016 update `Towards Our Greater Sydney 2056'.}
But there has been relatively little public discussion of the overall rationale for significantly different planning outcomes.
So it is not surprising that the NSW Government backed down on its plans to amalgamate some local councils after councils challenged these amalgamations in court.
It is unclear if the state government is proceeding with the proposal to allow the Greater Sydney Commission or the Planning Minister to replace a local council as the planning authority if the council fails to update its Local Environment Plan in line with the Greater Sydney Commission's District.%
	\footcite{Stevens-2017-Report-to-NSW-Premier-Housing-affordaibility}
And proposals to change Local Environment Plans in Sydney to enable greater development are under fire from within the NSW Government.%
	\footcite{OKeefe-2017-Threats-revolt-controversial-plans-increasing-Syd-pop}

\subsection{Independent panels should determine more development applications}\label{subsec:independent-panels-should-determine-more-development-applications}

Local councils tend to reflect the interests of existing rather than potential residents.
In order to reflect the broader public interests, some states have shifted responsibility for determining development applications from councils to independent panels.
These independent panels should reduce the workload for council staff, speed-up approvals, reduce the risk of corruption, and provide greater certainty for developers.
For example, the NSW Government recently announced that Independent Hearing and Assessment Panels (IHAPs) will be mandatory across all Sydney and Wollongong councils and will assess applications for developments valued at \$5 million to \$30 million.%
	\footnote{\textcite{NSWDPE2017ihaps}. The upper threshold was recently increased from \$20 million to \$30 million. A Sydney Planning Panel operates in each of the five districts in Greater Sydney and these panels will assess development applications with an investment value of more than \$30 million (\textcite{NSWDPE2017_sydney_planning_panels}).}
Other states should follow suit.

\subsection{State governments should increase density along key transport corridors}\label{subsec:states-should-increase-density-along-key-transport-corridors}

State governments should increase density along transport corridors,%
	\footcites{Adams-2010-Transforming-Aust-cities}{IA_2018_Future_cities}
which would both boost housing supply and use existing transport infrastructure better.

Boosting density along transport corridors could deliver a substantial amount of new housing in our largest cities.
For example, \textcite{Adams-2010-Transforming-Aust-cities} estimates that denser development along urban train, tram and bus routes in Melbourne could accommodate between 1~million and 2.5~million extra people at an average population density of between~200 and 400 people per hectare, primarily through 4-to-8 storey buildings.
A similar exercise for Perth found that medium-density development along just seven transport corridors could deliver between 94,500 and 252,000 dwellings.%
	\footnote{At a population density of between~60 and 160 people per hectare (\textcite{PropertyCouncil-etal-2013-Transforming-Perth}).}
Other authors have questioned whether such a strategy could deliver quite so many new dwellings, citing concerns that many transport hubs are located near old shopping strips with heritage facades.%
	\footnote{For example, \textcite[][12]{BuxtonEtAl2015}.
	It also assumes that transport corridors can absorb more patronage; for the vast majority increasing capacity is possible, and the cost will be justified if population densities are higher.}
But on any view, denser development along transport corridors would deliver more dwellings than at present.

To achieve greater density along transport corridors, the appropriate height of development (say between 4 and 8 storeys) needs to be determined up-front and declared to be `as of right'.
Clear principles need to be established that govern the transition to properties that run along the back boundaries of the designated development sites.%
	\footnote{For instance, \textcite{Adams-2010-Transforming-Aust-cities} specifies minimum rules for applicable streets, heritage, front and rear height limits, parking, setbacks, and access, among other factors.}
Governments should also use mixed-use zoning in these areas to enhance liveability.%
  \footnote{For example, the Central Park mixed-use development in the inner Sydney suburb of Chippendale has received positive reviews (\textcite{williams2016centralpark}), although there was community opposition to the development.}
And, to reduce community resistance, governments could make clear that the denser developments will be accompanied by new or improved transport services.

Higher-density development is occurring in some states.
As shown in \Vref{fig:ckc-apartment-completions} and \Vref{fig:gRLB-crane-maps}, the NSW Government has encouraged development near transport hubs in recent years, for example around Green Square, Parramatta, Wolli Creek and North Sydney,%
	\footcites{KentPhibbs2017Charts}{RLB2017CraneIndex}[][8,72]{NSW-DPE-2014-Plan-for-growing-Syd}[][70--71]{Gurran-etal-Politics-planning-housing-supply-England-HongKong} and higher-density development is planned for new transport infrastructure, notably along the North West rail link.%
	\footcite[][10]{NSWNorthWestRail2013}
In Melbourne, higher-density development has been more centralised, although there is some higher-density development at suburban train stations.%
	\footnote{For example, at Ormond and Moonee Ponds.}

\subsection{States should increase the supply of greenfield land and make it easier to develop greenfield housing}\label{subsec:states-should-increase-the-supply-of-greenfield-land-and-reduce-the-time-and-costs-to-supply-new-greenfield-lots}

New housing in greenfield developments on the fringes of our cities is another important part of the housing supply story.
But, as discussed in \Vref{subsec:the-limits-to-greenfield-housing-in-sydney}, limited release of land, slow planning approval processes, excessive infrastructure charges, fragmented ownership, and geographical constraints have increased the price of greenfield land and restricted the supply of greenfield housing developments, particularly in Sydney.%
	\footcites{Kendall_Tulip_2018_zoning}{GlaeserGyourko2017EconImplications}{HsiehEtAlSupply2012}{Urbis2011Housing}{Property-Council-2016-Delays-costing-new-homebuyers}{UDIA-201718-PreBudget-submission}

State and local governments are responsible for delivering new greenfield housing.
The precise reforms needed vary between states and local government areas.
A number of public inquiries have recommended reforms to reduce the cost and increase the supply of greenfield land.
These include proposals to:

\begin{itemize}
\item
  Introduce \textbf{housing codes for greenfield developments}, to speed-up greenfield developments.
\item
  \textbf{Maintain a long-term supply of new land for development} of around 15-20 years%
    \footnote{It can take up to ten years after rezoning commences before a subdivision of land is completed, infrastructure is installed and building can commence.
	If processes outside of planning are included, it can take up to 15 years between site assembly and building construction (\textcite[][125, 137]{PC2011PerformanceBenchmark}).
	Developers complain of a lack of \emph{serviced} land.}
  and align council housing targets with population forecasts and city-wide strategic plans.
\item
  \textbf{Tighten statutory time frames for re-zonings} and planning decisions.
  This would make the regulatory processes more disciplined and give developers a better idea of the time they should allow for each project.%
	\footcite[][XLIX]{PC2011PerformanceBenchmark}
  Where councils fail to meet statutory time frames, applications should be deemed to have been approved (as occurs with some applications in Queensland and the ACT).%
	\footcite[][82--83]{PC2011PerformanceBenchmark}
\item
  \textbf{Reform infrastructure charges} in line with the Productivity Commission's general principles on infrastructure costs.%
  	\footcite[][XLVI]{PC2011PerformanceBenchmark}
  This would involve levying charges on developers when local residents will primarily benefit from local public infrastructure such as parks and roads.%
	\footnote{Ideally, the incremental cost of local infrastructure attributable to each property would be reflected in developer charges.
    Infrastructure benefiting existing residents should be funded by user-charges where appropriate, or via general taxation (\textcite[][172]{Commission2014PublicInfastructure}).}
Infrastructure charges on developers should be set as close as possible to the cost of providing the local public infrastructure in new developments.%
	\footcites[][170]{Commission2014PublicInfastructure}[][18]{PC-2017-shifting-dial-potential-of-land}
Where councils aim to capture a share of windfall profits from rezoning or planning gain, this should be explicit and the charges should be reasonable and predictable, and only aim to capture a share of the economic value added above costs and a reasonable risk-adjusted return on capital.%
	\footnote{For example, \textcite{Terrill-Emslie-2017-Value-capture}.
	\textcites{Spiller-AndersonOliver-2015-Revisiting-economics-inclusionary-zoning}[][113]{Gurran_Bramley_2017_urban_planning_housing_market}[][10]{Terrill-Emslie-2017-Value-capture}
	note that developer charges are most likely to be borne by the landowner at the time the charge is determined by reducing the price a developer will be willing to pay for the land, particularly if the charges are known in advance.
	Yet developer charges are often poorly targeted at capturing value uplift since they are charged per property or per square metre of floor space, and tax some windfall gains but not others.}
\item
  \textbf{Use state government land organisations as the initial developer} in greenfield areas.
  These organisations could develop initial infrastructure in greenfield areas and provide a template for developers to follow.
  This would create a precedent for planning decisions and deliver initial infrastructure to greenfield areas, giving developers greater certainty about the prospects for a new greenfield development.%
	\footnote{\textcite[][137]{PC2011PerformanceBenchmark}.
    It was suggested in the NSW Government's housing affordability package that the government-owned developer, Landcom, take an active role to improve housing affordability.}
  By delivering a consistent supply of greenfield land, more active government-owned developers could also reduce the practice of `land banking'.
\item
  Require developers to \textbf{build a mix of lot sizes and housing types in new developments}.%
	\footcites{Kelly-etal-2012-Tomorrows-suburbs}{NSW-2015-Planning-Complying-date}
  Smaller lot sizes are more affordable and appeal to a different segment of the population than traditional detached homes.%
  \footcite[][129]{Treasury2013Housing}
  \Vref{tbl:Housing-stock-vs-preferences} shows that there is an undersupply of townhouses, units and apartments in the fringe- and outer-suburbs of Melbourne and Sydney.
  A diversity in lot sizes and house types will increase the flexibility of new suburbs as demographics and preferences change.
\end{itemize}

In addition to ensuring a steady supply of new greenfield land for development to house current generations, councils and state governments need to make greenfield developments more adaptable to meet the changing needs of future residents.
This could be done by putting a time limit on restrictive covenants; creating broader, mixed-use zones; regularly reviewing zoning; and creating an option for a majority of residents of a block to sell the entire block to a future re-developer.%
	\footcite{Kelly-etal-2012-Tomorrows-suburbs}

There are some moves in this direction.
The NSW Government is proposing to simplify development approvals in greenfield areas to speed-up supply.
And as part of its housing affordability package, released in March 2017, the Victorian Government is rezoning land for 100,000 new houses on Melbourne's fringe,%
	\footcite{VicStateGov2017Homes}
and funding a pilot program to speed-up the planning and zoning process for new greenfield developments.


\subsection{States should introduce betterment taxes to capture some of the windfall gains when land is rezoned}\label{subsec:states-should-caputure-windfall-gains-zoning}

Re-zoning land to allow more housing to be built in established suburbs, such as allowing low-rise apartments on a large suburban block, will make housing more affordable and is a key recommendation of this report.
But re-zoning also results in a windfall gain to the landowner as it increases the value of the land. State governments should introduce a betterment tax to capture some of the windfall gain from re-zoning, as the ACT Government does with its lease variation charge (see \Vref{box:ACT-lease-variation-charge}).

State governments and instrumentalities such as water authorities, currently use infrastructure charges on new developments as a way to tax some of the gains from re-zoning (for both greenfield and infill developments).
But infrastructure charges are generally tied to a particular piece of infrastructure and generally do not tax the full land value uplift from re-zoning.%
    \footcites[][11]{Terrill-Emslie-2017-Value-capture}[][10]{SGS2016__tech_paper_value_capture}
They also tend to be arbitrary, without a fixed basis for calculation related to the actual value of the zoning uplift.
Given the value created by individual zoning decisions, this creates significant opportunities for corruption.
State governments should introduce betterment taxes that explicitly capture most of any windfall gain from re-zoning, in combination with changes to state property taxes (see \Vref{subsec:replacing-stamp-duties-with-general-property-taxes}).

\begin{smallbox}{The ACT's `Lease variation charge'}{box:ACT-lease-variation-charge}

The ACT Government introduced a codified `Lease variation charge' (LVC) in 2011, replacing the `Change of use charge' which had operated since 1971.%
	\footcite[][4]{Macro_2010_ACT_change_of_use}
The LVC charges leaseholders for changes to their lease that allow a higher-value use of the land (all land in the ACT is leased from the government, unlike in states which have a freehold land title system). The LVC aims to capture some of the windfall gains that leaseholders receive from a beneficial change to their lease, such as permission to build higher density housing,%
	\footcite[][10]{SGS2016__tech_paper_value_capture}
so the LVC in effect acts like a betterment tax on the re-zoning of land.

The LVC aims to capture 75 per cent of the increase in value from a change to a lease.%
  \footcite[][10]{SGS2016__tech_paper_value_capture}
The amount payable by a leaseholder is codified for specific lease variations in each suburb, either on a per dwelling (for residential use types), or per floor area (for commercial uses) basis.
  \footcite{Prosper_murray_2016_ACT_land_value_tax}
The ACT Government codified the lease variation charge to provide greater certainty for developers and to simplify the administration of the scheme.

The ACT's unique leasehold land titling system enabled the implementation of this type of quasi-betterment tax. Other Australian jurisdiction could introduce an explicit betterment tax to achieve the same effect as the LVC.  

% http://www.planning.act.gov.au/topics/design-and-build/fees/change_of_use_charge_-_lease_variation_charge 
% http://infrastructureaustralia.gov.au/policy-publications/publications/files/SGS_Technical_paper_on_value_capture-September_2016.pdf 
% http://www.acilallen.com.au/cms_files/acgleasevariationcharge2012.pdf 
\end{smallbox}


\section{States should reform property taxes to improve housing affordability}\label{sec:state-governments-should-reform-property-taxes-to-improve-housing-affordability}

Two property tax reforms could improve housing affordability and increase economic growth:

\begin{itemize}
\item
  Replacing stamp duties with general property taxes would lead to more efficient allocation of the housing stock, reducing the total amount that people pay for their housing.
\item
  Applying land taxes at the same rate irrespective of a person's total property holdings would encourage more institutional owners of rental properties, leading to more of the long-term leases that many tenants want.
\end{itemize}

\subsection{States should replace stamp duties with general property taxes}\label{subsec:replacing-stamp-duties-with-general-property-taxes}

State governments should abolish stamp duties and replace them with a general property tax, as the ACT Government is doing.

Stamp duties on the transfer of property are among the most inefficient of taxes: they discourage people from moving to housing that better suits their needs so that the housing stock is used more efficiently.
Sometimes they discourage people from moving to better jobs.%
	\footnote{\textcite{hilber2017transfer} found that stamp duties on UK residential property strongly discouraged moving a short distance or for a better dwelling, leading to misallocation of dwellings in the housing market.
    But they found stamp duties were less likely to discourage a longer-distance move to take a new job.}

The effects of stamp duty are material: one study found that a 10 per cent increase in stamp duty can reduce housing turnover by 3 per cent immediately, and 6 per cent in the long run.%
	\footcite{Davidoff-Leigh-2013-How-do-stamp-duties-affect-the-housing-market}
The misallocation of housing stock is now obvious in Australia: spare bedrooms are much more prevalent in owner-occupied dwellings -- where housing moves are constrained by stamp duty -- than in the private rental market where they are not (\Vref{fig:owners-spare-bedrooms}).%
	\footnote{While an owner-occupier could move and avoid paying a second round of stamp duty by keeping their home and renting a new home, very few households do so (\textcite[][23]{RBA-Lealetal-2017-Housing-market-turnover}).
	This is not surprising given that home-ownership provides much more secure tenure than private rental in Australia (see \Vref{sec:renting-is-relatively-unattractive-under-current-policy-settings}).}

\begin{figure}
\caption{Owner-occupiers are more likely than renters to have multiple spare bedrooms}\label{fig:owners-spare-bedrooms}
\units{Households needing extra bedrooms or with spare bedrooms, 2015-16, per cent}
\includenextfigure{atlas/Charts-for-housing-affordability-report.pdf}
\sources{\textcites{ABS-201516-occupancy-and-costs}{Jericho-2017-theGuardian-If-homeownership-shrinks-retirement-system-in-trouble}.}
\end{figure}

The economic gains from stamp duty reform are large: a national shift from stamp duties to a broad-based property tax could leave Australians up to \$17 billion a year better off, according to estimates based on the excess burden of taxes.%
	\footnote{Updated from \textcite[][11]{DaleyCoates-2015-Property-taxes} using updated estimates of the excess burdens of taxes provided in \textcite{CaoHoskingKouparitsasEtAl2015}.}

The economic drag of stamp duties has increased over the past two decades.
Average rates of stamp duty have risen substantially in all states as thresholds have not kept pace with rising house prices (\Vref{fig:stamp-duty-rates}).
This is probably a material cause of housing turnover falling from 8 per cent a year in the early 2000s to below 5 per cent today.%
	\footnote{\textcites[][Graph~1]{RBA-Lealetal-2017-Housing-market-turnover}{Kusher-Core-Logic-2017-Dwelling-construction-surges}.
    It is difficult to disentangle the precise effect of stamp duties on turnover.
    Housing turnover rates might also have fallen because of an ageing population (older households move less), and lower rates of interstate migration.
    On the other hand, housing turnover rates would be expected to rise with higher rates of international migration, high price growth, and lower rates of home-ownership (\textcite{RBA-Lealetal-2017-Housing-market-turnover}).}

\begin{figure}
\caption{Effective rates of stamp duty have risen sharply in all states in the past two decades}\label{fig:stamp-duty-rates}
\units{Stamp duty payable on median-priced house in each capital city}
\includenextfigure{atlas/Charts-for-housing-affordability-report.pdf}
\noteswithsources{Median prices are for a detached house.
Darwin median price is for 2000. Assumes that the purchaser is not eligible for a concessional rate of stamp duty}%
{\textcite{Property-Council-2016-Stamp-duty-analysis}; Grattan analysis}
\end{figure}

Reducing stamp duties and increasing general property taxes would not affect housing prices much in the short run, but in the long run the prices of larger dwellings might reduce a little.
In the short run, dwelling prices would be largely unchanged: the boost to the purchasing power of prospective homebuyers as stamp duties were abolished would be offset by higher recurrent property tax bills, which would be capitalised into property values.%
	\footnote{Both stamp duty and a broad-based land tax would be fully capitalised into land values, in which case a stamp duty/land tax swap would be neutral with respect to house prices.
    See: \textcites{Coates2017PropertyTaxReformAdelaide}[][5]{Freebairn-2017-Reform-options-for-state-property-taxes}.}
However in the long run, a better allocation of the housing stock would lead to lower prices, particularly for larger dwellings.
Overall the average price of housing would fall a little.%
	\footnote{For example, \textcite{Abelson-2016-Housing-costs-policies} estimates that abolishing stamp duties in NSW could increase the effective NSW housing stock by up to 2 per cent, based on an analysis of unneeded spare bedrooms, reducing NSW house prices by 6 per cent.} %
Abolishing stamp duties may help some households by lowering the deposit hurdle where that is the constraint on purchasing a home. 

Proposals to switch from stamp duty to land tax have stalled because the politics are hard.%
    \footnote{Some states may also be discouraged from unilateral reform since any state moving first may be `penalised' by the way the GST sharing formula currently operates (\textcite[][100]{PC-2017-HFE-draft-report}), although it would be better off overall (\textcite[][8--10]{DaleyCoates-2015-Property-taxes}).}
Recent purchasers would be reluctant to pay an annual property tax so soon after paying stamp duty.
Meanwhile a property tax would pose difficulties for people who are asset-rich but income-poor, especially retirees who have limited incomes but own their own home.
And property taxes cause considerably more angst among voters than stamp duties because they are more salient: quarterly property tax bills are a far stronger reminder of the tax than stamp duties that are paid in full upon purchase, even though the stamp duty bill is much larger.%
	\footnote{For example, \textcite{CabralHoxby2012} find that American jurisdictions where property taxes are built into mortgage repayments -- known as tax escrow -- tend to have higher average property tax rates than jurisdictions where property owners pay the tax directly.} 

The right design for a property tax to replace stamp duty can help overcome the politics.
Rather than jacking up existing land taxes -- which exclude more than half of all land by value, especially owner-occupied housing -- state governments should fund the abolition of stamp duties through a property levy imposed via the council rates base.%
	\footnote{Rates are applied to all properties within a council area with few exemptions.
	There are no exemptions for owner-occupied housing or agricultural land, and constant rates apply from the first dollar of property value with no minimum threshold.
	The largest exemption from council rates is for some non-profit, non-government organisations such as charities, schools and public hospitals. \textcite[][16]{DaleyCoates-2015-Property-taxes}.}
The property levy could be applied to only the unimproved value of land, or to the combined value of land and buildings. States should adopt whatever tax base is already used for council rates in their jurisdiction.%
	\footnote{Although a levy on unimproved value is theoretically better, the practical impacts on investment of a levy on capital-improved values would be small. For example, with a 0.3 per cent property tax on land and buildings, a landlord doing capital improvements of \$100,000 would
	need to collect a mere \$24 extra a month in rent to recoup the
	costs of the tax: \textcite[][Box~2]{DaleyCoates-2015-Property-taxes}.} 
An annual flat-rate tax on unimproved land values of between 0.5 and 0.7 per cent of property value would be sufficient to replace stamp duties in each state. Alternatively, a levy on capital-improved property values of roughly half that rate would be sufficient to fund the abolition of stamp duties in each state.%
	\footnote{The precise tax rate required to replace stamp duties in each state depends on the stamp duty regime and the rate of housing turnover, as well as potential impacts on the distribution of GST revenues among all states. \textcites{Coates2017PropertyTaxReformAdelaide}[][8]{DaleyCoates-2015-Property-taxes}.} 
And replacing stamp duties with a progressive property levy calculated separately for each individual land plot -- as the ACT has done -- could minimise the windfall gains to larger home-owners from the swap, especially if the property levy also funds the abolition of progressive state land taxes.%
    \footnote{\textcites[][]{Coates2017PropertyTaxReformAdelaide}[][15]{Freebairn-2017-Reform-options-for-state-property-taxes}.}

The right reform design can also help manage the transition from stamp duty to a property tax.
A gradual transition to a broad-based property tax such as that adopted in the ACT is best:%
    \footnote{The ACT Government is already five years into a 20-year plan to replace stamp duties with broad-based property taxes.
    Annual general property rates on a family home on land worth \$500,000 have increased from roughly \$2,200 a year in 2012 to \$3,000 just four years later.
    At the same time, the stamp duty on a home worth \$500,000 has fallen by more than five times that amount: from \$18,050 to \$13,460. \textcite{DaleyCoates-2016-AFR-Following-ACT-landtax-boosts-growth}.}
it would provide a stable revenue stream while delaying the full impost on those who recently paid stamp duty.
Over time the property tax would hit asset-rich, income-poor households.
That's why state governments should allow asset-rich, income-poor households to stay in their homes, by allowing them to defer paying the levy until they sell their property.%
	\footnote{Deferral arrangements are already available for seniors paying council rates in South Australia, Western Australia and the ACT\@ (\textcite[][20]{DaleyCoates-2015-Property-taxes}).} 

Alternative proposals to grandfather existing home-owners from any recurrent property tax until the property is next sold,
    \footcite{McKellInstitute2016APlantoEndStampDuty}
or that allow purchasers to choose between paying stamp duty or land tax,%
    \footcite[269][]{HenryTaxReview2010-Part2-Detailed-analysis}
would fully exempt asset-rich, income-poor households from paying the levy unless they chose to move. These options would also neutralise perceptions of unfairness among those who have recently paid stamp duty.
However, both options pose significant threats to state budgets, because the state foregoes stamp duties received up-front in favour of a much smaller recurrent property tax paid each year.
Such a shortfall could be financed,%
    \footcite{PBO-2016-Financing-state-property-tax}
but would still show up as a large deterioration in states' headline budget balances.%
	\footcite{Coates2017PropertyTaxReformAdelaide}

\subsection{States should reform land taxes to encourage institutional investment in rental housing}\label{subsec:flattening-land-taxes}

Reforms to existing state land taxes could also encourage more institutional investors into the private rental market, thereby improving security of tenure for renters.

As discussed in \Vref{subsec:state-government-land-taxes-make-short-term-leases-common-and-contribute-to-insecurity-of-tenure}, land taxes are levied on a progressive scale so that people with larger land holdings pay a higher rate of land tax per dollar value of land owned.
In addition, no tax is levied on people with total landholdings less than a threshold.
These tax-free thresholds range from \$25,000 in Tasmania to \$600,000 in Queensland.%
	\footcite{NSW-Treasury-2016-Interstate-comparison-taxes-201516}

Progressive land taxes levied on total landholdings and generous tax-free thresholds discourage larger landholdings and largely explain why small investors dominate Australia's rental housing market.%
    \footnote{The interaction of a fifty per cent capital gains tax (CGT) discount with negative gearing distorts investment decisions also favours mum-and-dad investors as any losses can be deducted in full but investors are only taxed on half the capital gains. Institutional investment in private rental accommodation is more common overseas (see \textcite[][Table~7]{AHURI_2018_private_rental_housing_Martin_etal}).}
Institutional investors are probably more willing to offer long-term leases to tenants, for two reasons: they are less likely to face cash-flow problems or the need for portfolio diversification that can force sales by small-scale investors; and they can pool the risk of leasing any one property to a bad tenant across the many properties owned. Consequently, institutional investors would be less likely to be put off by stronger tenancy laws that provide renters with more secure tenure (see \Vref{sec:states-governments-should-make-renting-a-more-attractive-option-by-changing-tenancy-laws}).

Unlike mum-and-dad investors, institutional landlords should also be able to use economies of scale in managing and maintaining rental properties (such as renovations and repairs, and finding tenants) to reduce costs and improve the quality of service provided to tenants.%
	\footcites{PwC-2017-A-fixed-abode}{Freebairn2016Housing}[][417]{HenryTaxReview2010-Part2-Detailed-analysis}
Institutional landlords may also provide a better service to tenants because they are in the business of leasing out properties and have a brand to protect.%
	\footnote{According to \textcite{Irvine-2016-SMH-Hidden-tax-hurts-renters}, `larger-scale property owners would have a bigger incentive to manage their tenants' needs in a timely and professional manner to maintain their reputation and attract good tenants'.
	See also \textcite{Duke-2017-Domain-Speculative-investors-cause-rent-pain}. However, \textcite{AHURI_2018_private_rental_housing_Martin_etal} caution that some institutional landlords have been accused of treating tenants poorly.}

Governments and property observers are keen to talk about and encourage institutional investment in residential housing, especially the `build-to-rent' sector,%
    \footnote{The NSW Government established a build-to-rent taskforce in August 2017.} 
but often fail to mention the role land taxes play in discouraging this type of housing.%
    \footnote{\eg~\textcite{EY-2017-multi-family-asset-class-australia}.}
Institutional investors are unlikely to enter Australia's residential housing market in significant numbers unless large and small residential property investors are treated more equally under state land tax regimes.%
    \footnote{One exception may be tall residential towers -- such as 10 storeys or higher -- where land accounts for only a small share of the cost of constructing each apartment 
    (\textcite{Ahlfeldt-McMillen-Vox-2017-Skyscrapers-and-land-values}). 
    However such towers are only ever likely to account for a small share of residential rental housing.} 

The precise reforms to state land taxes will depend on states' existing land tax regimes and require a detailed assessment of the budgetary and social implications. However, a number of broad reform options can be identified.

The simplest way to reform state land taxes would be to shift to a progressive land tax assessed on the value of each property owned, rather than on the combined value of an owner's total landholdings.%
	\footnote{Recent Commonwealth and state tax reviews have also considered levying land tax with higher tax rates for land with a higher value 
	(\textcites[][265]{HenryTaxReview2010}[][41]{GovernmentSouthAustralia2015-State-Tax-Review-Discussion-Paper}), but the problems with progressive rates probably outweigh the benefits (\textcite[][18]{DaleyCoates-2015-Property-taxes}).} 
A new revenue-neutral progressive land tax regime could be designed to most closely match the tax liabilities paid by existing landowners in each state, thereby minimising the windfall gains and losses from any reform. Such a land tax regime would provide incentives to assemble a portfolio of multiple strata-title rental properties, but would still discourage investment in the build-to-rent sector since the entire building would be assessed as a single site and taxed at the top marginal land tax rate.%
	\footcite[][16]{Freebairn-2017-Reform-options-for-state-property-taxes} 

Alternatively, progressive land tax rates could be flattened, and tax-free thresholds abolished, and replaced with a flat-rate land tax applying from the first dollar of land value. Such a reform would remove the land tax hurdles to institutions investing in \textit{either} strata-title or built-to-rent housing, since both investments would be taxed at the same low rate as smaller investors.%
	\footnote{For example, the top marginal land tax rate in NSW is 2 per cent, whereas \textcite[][12]{Freebairn-2017-Reform-options-for-state-property-taxes} estimates a flat land tax of around 0.2 per cent on the existing land tax base would be sufficient to fund the switch.} 

Separate land tax schedules could be introduced for residential and commercial land, with residential land paying a low flat rate and commercial land remaining subject to a progressive land tax schedule in order to to prevent windfall gains to large existing commercial landholders.

\section{State governments should make renting a more attractive option by changing tenancy laws}\label{sec:states-governments-should-make-renting-a-more-attractive-option-by-changing-tenancy-laws}

As housing becomes less affordable to own, more Australians will inevitably remain renters for longer, and a growing number will be destined to rent for their whole lives.

Yet renting is relatively unattractive, given current rental markets and policy settings.
As noted in \Vref{sec:renting-is-relatively-unattractive-under-current-policy-settings}, renting is generally much less secure; many tenants are restrained from making their house their home; and tenants miss out on the tax and welfare benefits of home-ownership.%
	\footcite[][19--21]{KellyHarrisonHunterEtAl2013}
Renters are~forced to move much more often than home-owners, and are less satisfied with their housing.

State governments should make renting more attractive by changing residential tenancy laws to increase the security of renters and help renters make their property feel like their home.

Worthwhile changes include:

\begin{itemize}
\item
  \textbf{Removing `no grounds' evictions} by clearly prescribing grounds for termination.%
  \footcite{KellyHarrisonHunterEtAl2013}
\item
  \textbf{Extending minimum notice periods} that apply when landlords terminate a lease. Landlords can terminate leases on grounds such as moving in themselves or selling the property, generally with 30-to-60 days notice.%
	\footcite{Hulse-etal-2011-AHURI-Secure-occupancy-rental-housing}
\item
  \textbf{Creating a different regime for long-term leases}, such as those of five years or more, which, in exchange for more security of tenure, could shift responsibility for some maintenance and minor repairs to tenants.\footnote{Tenancy laws require  landlords to ensure rented premises are provided fit for habitation and maintained in a reasonable state of repair, except in Tasmania where the property must be maintained in the condition when the lease began (\textcite{Martin2017renting}).}
\item
  \textbf{Increasing tenants' freedom} to make their house their home, by allowing them to own pets and to make minor modifications such as hanging pictures.%
	\footcites{KellyHarrisonHunterEtAl2013}{Vic-makingrentingfair}
\item
  \textbf{Increasing the transparency} of `bad tenants' lists, so tenants who are on such lists know why, and can seek to clear their name.
  \footcites{Irvine-2016-SMH-Hidden-tax-hurts-renters}{Natl-Shelter-2017-Life-in-Aust-private-rental-market}
\end{itemize}

Many other countries have some or all of these settings in place.
By comparison with many developed countries, Australia has standard terms that are significantly more favourable to landlords than tenants (\Vref{fig:rental-conditions-global}).\footcite[][8]{Hulse-etal-2011-AHURI-Secure-occupancy-rental-housing}
The Victorian Government recently moved to tip the balance more towards tenants.%
    \footnote{The major proposed changes include: abolishing `without grounds' evictions for ongoing leases; allowing tenants to keep pets and make minor modifications to the property unless the landlord has a reasonable reason to refuse; faster bond repayments; and only allowing rent increases every 12 months instead of every six months (\textcite{Vic-makingrentingfair}).
    Most proposals require amendments to the \emph{Residential Tenancies Act 1997} (Vic).}

\begin{figure}
\caption{Typical rental conditions vary around the world}\label{fig:rental-conditions-global}
\includenextfigure{atlas/Charts-for-housing-affordability-report.pdf}
\noteswithsource{This figure does not take into account the Victorian Government's proposed tenancy law changes}{\textcite[][Figure~3.1]{KellyHarrisonHunterEtAl2013}}
\end{figure}



Such changes in tenancy laws in favour of renters could reduce the supply of rental housing and increase rents, but the effects are likely to be vanishingly small. Recent years have shown there is no lack of investment capital available for housing investment. And in urban housing markets with tight constraints on new housing supply, almost all the impact will be on residential property prices rather than on rents.%
	\footcite{DaleyWood2016-Negative-Gearing-CGT}

Some may be concerned that property investors will sell their properties if tenancy laws are changed.
But this will have no discernible impact on rents.
If another property investor buys the property, then there is no change.
And if a person who is currently renting buys the property, then there would be one less rental property, but also one less renter.
There would be no change to the balance between supply and demand of rental properties in the short term, and hence no expected change in rents.\footnote{Economic theory suggests that stronger tenancy rules will reduce the long term supply of housing. But as noted in \textcite[][31]{DaleyWood2016-Negative-Gearing-CGT}, with tight constraints on supply of land suitable for urban housing, any impact would likely be very small.}
Changes to tenancy laws may result in some landlords becoming more selective when choosing tenants, which may increase vacancy rates. But this effect will likely be small. It is also possible that some landlords will use platforms such as Airbnb to rent out their properties on a short-term basis to avoid being covered by tenancy laws.

This is consistent with experience abroad.
In 2004 Ireland moved, from arrangements similar to Australia's, to increase security of tenure for renters.
The standard lease moved from 6-12 months to a legally prescribed six years, although landlords and tenants can terminate a lease in the first six months with 28 days' notice.
Thereafter, landlords can only terminate the lease on more narrowly prescribed grounds.
Notice periods increased in line with the length of the tenure.
The effect of these changes was obscured by the global financial crisis, but they did not obviously reduce the supply of private rental housing. Since the reforms were introduced in 2004, the Irish private rental sector has grown substantially as a proportion of all housing.
As in Australia (and Germany), the Irish rental market is dominated by small individual investors.%
	\footcite[][21]{KellyHarrisonHunterEtAl2013}

The effects of these proposed reforms on rates of home-ownership are hard to predict.
Some rental investors may choose to sell their properties to owner-occupiers if they feel they have less control over their investment.
But increasing the power of renters would also make renting more attractive relative to home-ownership, leading some prospective homebuyers to rent instead of buying.
Changing tenancy laws may also shift the prevailing social attitude that renting is inferior to owning a home.

\section{The Commonwealth Government should act to increase the supply of housing}\label{sec:the-commonwealth-government-should-act-to-increase-the-supply-of-housing}

\subsection{Reasons for Commonwealth involvement}\label{subsec:reasons-for-commonwealth-involvement}

Although the Commonwealth does not control the supply of housing directly, there are good reasons for it to provide incentives for the states to do so.
Coordinating action by the states is worthwhile because improved housing supply in one state spills over into lower prices in other states.
And the Commonwealth tax base is more likely than the state tax base to capture the increased revenues that flow from higher economic growth as a result of better housing supply.

Australia's housing markets are interconnected.
If, for example, only the Victorian Government substantially boosts housing supply, any improvement in affordability will be dispersed across Australia as residents of other Australian cities move to Melbourne, attracted by lower house prices relative to other major Australian cities.%
	\footnote{\textcite[][51]{Abelson-2016-Housing-costs-policies}.
	Similarly, \textcite{Aura-Davidoff-2008-Supply-constraints} find that loosening regulatory constraints on supply in an individual city would have little effect on house prices, whereas a coordinated boost to housing supply across major cities could result in large price falls.
	\textcite[][Table~E.6]{GeographicLabourMobility}
	modelling estimates that a 10 per cent differential in house prices between two regions increases migration between them by 1.8 per cent.}
Housing affordability in Melbourne would not improve much, nor would economic output per capita -- and infrastructure pressures would increase.
But because Australia's migration intake is largely determined by the Commonwealth, independently of state planning policies,
affordability would improve in other states, even though they would have avoided the political costs of increasing housing supply. 

\subsection{The Commonwealth should provide incentives to state and local governments to increase housing supply and abolish stamp duties}\label{subsec:the-commonwealth-can-provide-incentives-to-state-and-local-governments-to-increase-housing-supply-and-abolish-stamp-duties}

The Commonwealth should provide incentives to state and local governments to increase the supply of housing in good locations.%
    \footnote{For example, see \textcite{Deloitte2016-Fed-Incentives-Housing-Supply}.}

Under the National Competition Policy reforms of the 1990s, the Commonwealth Government provided financial incentives to the states.%
    \footnote{\textcite{PC-2005-Review-Natl-Competition}.
    A total of \$5.7~billion was allocated for payments from 1997-98 to 2005-06.}
The Commonwealth Government plans to use a new intergovernmental housing agreement and City Deals to encourage state and local governments to boost housing supply by offering incentive payments to support planning and zoning reform.%
	\footcites{Budget1718-Housing-pkg-Western-Sydney}{City-Deal-performance-framework}
These plans sound as if they are headed in the right direction, but they may well not deliver.
It isn't obvious that the Commonwealth can put enough money on the table to get states to make the politically difficult decisions on planning reform.
And in the case of City Deals, the process could still be derailed if agreements are motivated by a chase for votes in marginal seats rather than meaningful reforms.%
	\footcite{OBrien-2016-ABC-Townsville-stadium}
The signs aren't promising: the Commonwealth has signed City Deals for Townsville and Launceston and is working on City Deals for western Sydney, Darwin, Hobart and Geelong; yet Melbourne, with Australia's fastest population growth, is conspicuously absent.

The Commonwealth Government could also provide incentives to encourage state governments to abolish stamp duties and replace them with a general property tax.
A recent COAG agreement to encourage states to enact economic reforms is a step in the right direction. But again, the Commonwealth incentives seem too small to change the political calculus for difficult reforms.%
	\footcite{COAG-2016-Competition-productivity-enhancing-reforms}
	
A robust deal between the Commonwealth and state governments to boost housing supply would likely require a new agency to evaluate states' progress on housing and planning policy reform, similar to the former National Housing Supply Council, but as an independent statutory body with its own dedicated staff.%
    \footcite{NHSC}
Such an agency would be responsible for collecting and analysing data from state governments on housing completions and the performance of state and local government planning systems.%
    \footnote{\textcite{PC2011PerformanceBenchmark} established a range of metrics of state land use planning systems, including the overall time taken to complete developments, development approval timeframes, and the number of planning applications rejected or referred to administrative tribunals.
    These could be adopted as part of a performance reporting framework for any Commonwealth-state agreement on boosting housing supply.}
Ideally such an agency would also publish independent research on aspects of housing affordability, building on the model of institutions such as the Parliamentary Budget Office.


\subsection{The Commonwealth is taking some action to increase the supply of greenfield land}\label{subsec:the-commonwealth-is-taking-some-action-to-increase-the-supply-of-greenfield-land}

The Commonwealth Government can do a little to increase the supply of greenfield land available for residential development.
The 2017 Budget created a Commonwealth land registry. Members of the public will be able to view the registry and suggest alternate uses of Commonwealth land.
The Commonwealth Government also announced that it will develop surplus Defence land in Maribyrnong in Melbourne.
But the overall quantity of land is small relative to population growth.%
	\footnote{The Commonwealth Government estimates that this land will be sufficient to accommodate up to 6,000 new homes (\textcite{Budget1718-unlocking-CW-land}). However the land will not be released for sale for several years due to remediation works to remove toxic contaminants from the site (\textcite{TheAge-2017-Housing-fix-still-years-away}).}
The Commonwealth also indicated that it intends to make the new National Housing and Homelessness Agreement conditional on the states increasing the supply of greenfield land through planning reforms.%
	\footcite{Budget1718-New-Natl-Housing-Homelessness-Agreement}
But this lever is unlikely to be politically feasible: it will be hard for the Commonwealth to delay funding for social housing on the grounds that the states have failed to change laws relating to general land release.\footnote{The move was instrumental in leading State Treasurers to set up their own body, independent of the Commonwealth, see \textcites{Tabakoff_2017_Aus_revolt_against_Canberra}{NSW-Treasurer-press-release-States-treasurers}.}

\section{All governments should improve transport networks to increase the effective supply of well-located housing }\label{sec:all-governments-should-improve-transport-networks-to-increase-the-effective-supply-of-well-located-housing}

Governments also need to improve transport networks, by using existing transport infrastructure more efficiently and only building more effective transport projects.\footcite{IA_2018_Future_cities} To maximise the impact on housing affordability, improvements in transport infrastructure need to be paired with changes to land use planning rules to boost the supply of new homes.%
    \footcite{Schmahmann-2016-unlockingcitychapingpotential}

First, state governments should consider introducing \textbf{congestion charging}.%
	\footcites{OrangeBook-2016}{Fletcher-2016-Sydney-Institute-speech}{Terrill-2017-Road-congestion}
Charging drivers a fee to drive on congested roads would lessen the worst effects of congestion and enable roads to be used more efficiently.
A congestion charge needs to discourage only a small proportion of people from driving to enable a big increase in traffic speed.%
	\footcite[][172]{KellyDonegan2015-City-limits}

Second, \textbf{governments need to improve how they decide on transport infrastructure investments}.
Commonwealth and state governments have spent unprecedented sums on transport infrastructure in the past decade.
But often, they have not spent wisely.
Governments have tended to favour projects in swing states and marginal seats, rather than projects with the highest benefit-cost ratios.
Since June 2012, \$3.7~billion of Commonwealth money has been committed to transport infrastructure projects without a published evaluation, and a further \$2.6~billion before the proposals were submitted to Infrastructure Australia.%
	\footcite[][18]{Terrill-etal-2016-Cost-overruns-in-transport-infrastructure}
Governments should commit money to a transport infrastructure project only if Infrastructure Australia or another independent body has assessed it as a high priority and the business case has been tabled in parliament.
Governments should also consider the likelihood of cost overruns when assessing or announcing an infrastructure project.\footcite{Terrill-etal-2016-Cost-overruns-in-transport-infrastructure}

These reforms could make housing more affordable. Better functioning transport networks in our major cities would increase the supply of well-located land by making it easier for residents to access jobs across a larger share of the city from a given location. Therefore improving transport networks would see Australians obtain better located housing for a given price, even if actual market house prices don't actually fall.% 
    \footnote{Reducing transport costs (including travel time costs), or otherwise improving the amenity of a neighbourhood, effectively improves the quality of the existing housing. (\textcites[][6]{Abelson-2016-Housing-costs-policies}[][7]{Terrill-Emslie-2017-Value-capture}).} %
And better transport networks could also reduce the relative price premium for scarce inner-city land by making fringe suburbs a marginally more attractive alternative to established suburbs closer to CBDs.%
    \footcite[][25]{HousingAus17}

But the impact of improving transport networks on housing affordability should not be overstated. 

The areas where it would be practicable to implement congestion charging remain limited: congestion charges only make sense where there is congestion.%
    \footnote{\textcite[][41]{Terrill-2017-Road-congestion} notes that when roads are not congested, the charge should be zero, because a driver using the road at that time does not slow anybody down.} 
And cordon based congestion charging schemes -- likely the best design for major Australian cities%
    \footnote{For example \textcite[][41]{Terrill-2017-Road-congestion} recommends the Victorian Government investigate a ``cordon'' scheme for Melbourne that encompasses key arterial roads in inner suburbs as well as the CBD\@.}\space
-- could actually make inner city homes relatively more expensive, as it has in London.%
    \footnote{\textcite{Tang-2015-Traffic-Externalities-Housing-Prices} finds that the introduction of congestion charging in London saw prices inside the congestion cordon rise 4 per cent compared to those outside the cordon as home-owners were prepared to pay a premium to live in inner city areas where the congestion zone applies.} 

Meanwhile improving the quality of project selection may not lead to significantly more transport infrastructure than we have currently. 
After all, the vast bulk of the transport infrastructure we will use over the next 20 years has already been built -- therefore new additions to the stock are always small. Nor is it clear that Australia faces a substantial transport infrastructure deficit, as often claimed.%
    \footnote{For instance, Engineers Australia regularly calls for major changes on the basis of a qualitative assessment (\textcite{Australia2010a}).
    Infrastructure Australia estimated the deficit at \$300 billion (\textcite[][6]{Australia2013}).
    The Reserve Bank Governor spoke recently on how Australia's ``underinvestment'' in transport infrastructure had pushed up house prices (\textcite{Lowe-2017-Speech-RBA-Dinner}).
    Yet the evidence and methodologies to substantiate such claims are not convincing (\textcite{TerrillCoates-2016-infrastructuredeficit}).} 
In fact, based on our track record over the last decade, taxpayers could end up both better off, and spending less on new transport infrastructure than they do now.

The dominant rationale for these reforms would instead be their economic and budgetary benefits. The economic costs of congestion are very large -- estimated at around \$16 billion a year nationwide, and are projected to double by 2030.%
    \footnote{\textcite[][1]{BITRE-2015-congestion-costs} estimates that congestion is costing \$6.1 billion a year in Sydney and \$4.6 billion a year in Melbourne. Infrastructure Australia (IA) says that congestion cost \$5.5 billion in Sydney and \$2.8 billion in Melbourne in 2011, with these costs projected to increase to \$14.8 billion and \$9.0 billion respectively by 2031. However mitigating congestion is not costless (\textcite[][12--13]{Terrill-2017-Road-congestion}).}
And avoiding wasteful spending on bad transport projects where the costs exceed the benefits would save Commonwealth and state government budgets billions of dollars each year -- funds that could be allocated to funding better transport projects that would actually produce a positive return for the community, or to other spending priorities.

\section{Improving affordability for low-income Australians}\label{sec:helping-the-bottom}

As noted in \Vref{sec:what-this-report-does-not-do}, this report does not attempt to analyse comprehensively the provision of `affordable' housing such as public and community housing (subsidised housing that is provided specifically for low-income earners).%
     \footnote{There are a number of types of affordable housing, ranging from sub-market private rental housing provided at 75-to-80 per cent of the market rate, through to public and community housing where rents are more heavily subsidised and usually set at 25-to-30 per cent of tenants' incomes.}
This topic deserves more detailed consideration than this report can provide.
But there are some lessons for affordable housing that emerge from our analysis of the general housing market.

\subsection{There are genuine issues for low income earners}\label{subsec:there-are-genuine-issues-for-low-incomes}

This report shows that housing affordability has become much worse for low-income Australians than for the population as a whole (\Vref{sec:rising-house-prices-have-widened-inequality-within-generations}).
Low-income households are spending more of their income on housing, their rents are rising faster than their incomes, the price of housing they might buy is increasing particularly quickly, their home-ownership rates are falling particularly quickly, and their levels of financial stress are rising faster. 
Many are being geographically segregated into housing on the edges of our cities where employment prospects and social outcomes are worse.

So there is a powerful case for additional public support to help those worst-off to cope with rising housing costs.
Obviously the `right' degree of redistribution within society is a value choice that is contested.%
    \footcite[][9]{DaleyCoatesWood-2015-Super-tax-targeting}  
Previous Grattan work has focused on the impact on the bottom 20 per cent of the income distribution – generally those who are worst off.
This reflects a consensus in Australian political culture that policy should assist those who are less well-off to have opportunities to pursue lives that they have reason to value.%
    \footnote{Many argue that policy should also aim to distribute resources more equally. See: \textcite[][21]{DaleyEtAl-2013-BalancingBudgets}.}%
    
\subsection{Increasing social housing subsidies is particularly important for those at high risk of long-term homelessness}\label{subsec:Increasing-social-housing-subsidies-is-particularly-important-for-homelessness}

It may well be that the most important role of social housing is to provide secure housing for those under severe stress, at significant risk of becoming homeless for the long term.
Many homelessness programs now adopt a `housing first' strategy.%
    \footnote{\eg~\textcite{Mission_Australia_2014_homelessness}.}
    
But with minimal additions to the total number of social housing dwellings, this strategy is proving difficult.
Existing tenants tend to have `squatters rights' to stay because it usually proves politically impossible to require an existing low-income tenant to leave, even if the aim is to free up a place for someone else who needs the housing even more.
As a result there is little `flow' of social housing available for people whose lives take a big turn for the worse.
The crucial issue for social housing policy, therefore, may be to invest enough to ensure that there is a material increase in the total volume from year to year, so that there is always some availability for those at high risk of long term homelessness.

\subsection{Publicly funded social housing is unlikely to help most low income earners}\label{subsec:Social-housing-is-unlikely-to-help}

Funding to increase the volume of social housing stock will help those low income households who move into it.
But no plausible quantity of funding will be enough to provide subsidised housing for all of the 20 per cent of households typically classified as low income.
Even if the social housing stock is returned to its historical share of around 6 per cent of the total stock, by definition it will still house less than a third of households in the bottom 20 per cent.
Boosting the stock of social housing by 100,000 dwellings -- broadly sufficient to return the total affordable housing stock to its historical share of the total housing stock -- would require additional ongoing public funding of around \$900~million a year.% 
    \footnote{\textcites{Coates-Wiltshire-2018-InsideStory-conventional-wisdom-wrong}[][8]{Daley-etal-2017-Submission-Natl-housing-finance}.  Or a one-off upfront capital contribution of \$18 billion (assuming a 5 per cent discount rate). \textcite[][14]{Council-Fed-Fin-Relations-2016-Innovative-models-to-improve-supply-affordable-housing} estimated that the rental stream from social housing covers only 40 per cent of the costs of land, and building and maintaining \textit{social housing}.
    For a social housing dwelling -- where the tenant's rent is set at 25 per cent of income -- the funding gap is \$8,850 a year.
    Alternatively, boosting the supply of affordable housing (where rent is set at 75 per cent of market rent) by 100,000 dwellings would cost \$310 million a year based on estimates that the annual public subsidy required is around \$3,100 a year.
    While less costly to government, affordable housing inherently provides less benefit, or subsidy, than social housing.}
    
Therefore, governments need to pursue the reforms set out in this report that will improve housing affordability more generally.
Making housing cheaper overall will help low-income earners (\Vref{subsec:additional-housing-was-primarily-above-average-quality-although-this-is-less-of-a-problem}) -- and help more of them than increasing the social housing stock.
And these reforms will also reduce the amount of public subsidy needed to bridge the gap between development costs and what low-income earners can afford to pay.

\subsection{A social housing bond aggregator may modestly increase the supply of social housing}\label{subsec:a-social-housing-bond-aggregator-may-modestly-increase-the-supply-of-social-housing}

The Commonwealth Government announced in the 2017 Budget that it will establish a National Housing Finance and Investment Corporation to operate a `bond aggregator' for the social housing sector.%
	\footcite[][169]{Budget2017-18-BP2}
The corporation will borrow on behalf of community housing providers, and on-lend to the providers -- giving them access to cheaper and longer-term finance.%
	\footnote{In Australia the community housing sector typically relies on shorter-term bank debt (typically 3-5 years) (\textcite[][8]{EY-2017-multi-family-asset-class-australia}).}

The proposed social housing bond aggregator \emph{could} significantly improve housing affordability by boosting the supply of social housing, but only if it were paired with large ongoing public subsidies for social housing, at substantial cost to government budgets.
At the scale currently envisaged, and given the current economics of building social housing, it is unlikely to make much difference.%
	\footcite{Daley-etal-2017-Submission-Natl-housing-finance}

\subsection{Social housing could be boosted through inclusionary zoning}\label{subsec:Social-housing-and-inclusionary-zoning}

Governments are also exploring funding more social housing through inclusionary zoning.
This has become increasingly popular in Australia, in part because it might provide more affordable housing at no direct cost to government budgets.

Most state and some local governments have adopted some form of inclusionary zoning policies.%
    \footcite[][8--9]{Daley-etal-2017-Submission-Natl-housing-finance}
These policies come in a variety of shapes:

\begin{itemize}
    \item
    Governments may require new developments to contain a proportion of `affordable housing' that can be rented out at below-market rates.
    \item
    Governments can make it a condition of approval that a development includes a proportion of `affordable housing'. 
    \item
    Governments can give developers additional planning concessions, such as higher height limits or other bonuses, if they include some affordable housing in the development. 
    \item
    As a condition of approval, governments can require developers to pay a levy that funds the provision of affordable housing.
\end{itemize}

Apart from effectively increasing the subsidies for low income housing, inclusionary zoning is also seen as a way to encourage neighbourhoods with a greater range of incomes. Australia's capital cities are increasingly segregated: incomes in outer suburbs are lower, and growing more slowly (\Vref{subsec:higher-house-prices-are-contributing-to-a-greater-divide-between-the-have-and-have-nots-in-our-cities}).
It is arguable that more diverse neighbourhoods contribute to political and social cohesion because more people see first-hand `how the other half lives'. 

But there are risks with inclusionary zoning.  It may increase rents in the private rental market a little.%
	\footnote{Since the supply of new housing in Australian cities is relatively unresponsive to demand because of land use planning rules, the main impact of inclusionary zoning should be to reduce land values as developers are not willing to pay so much for developable land.
	Therefore in large part inclusionary zoning acts as a de facto tax on planning gain that captures some of the windfall to landowners when land is re-zoned.
	But since housing supply responds at least a little to prices (\ie~it is not perfectly inelastic) some portion of the costs will be reflected in higher rents in the private rental market (\textcite[][9]{Daley-etal-2017-Submission-Natl-housing-finance}).}
Those who are allocated affordable housing will be much better off; other low-income earners may be a little worse off. And if rules around inclusionary zoning are not clearly codified then ad hoc approaches which give great discretion to local governments increase the risk of corruption. 

\subsection{Housing support for low income earners is better provided as rent assistance}\label{subsec:Housing-support-for-low-income-earners}

A low income household that is allocated social housing somewhat arbitrarily receives much larger public benefits than other low income households.
Those in public housing (often for historic reasons) receive a much greater average level of assistance than Rent Assistance provides to private renters.%
    \footcite[][605]{HenryTaxReview2010-Part2-Detailed-analysis}

Beyond ensuring a flow of additional social housing for those at risk of long-term homelessness, it is arguable that further support for low-income housing should be focused on direct financial assistance for low-income renters rather than building more social housing.

\textcite[][491]{HenryTaxReview2010-Part2-Detailed-analysis} recommended that Rent Assistance be increased `so that assistance is sufficient to support access to an adequate level of housing'.%
	\footnote{Rent Assistance is designed to assist low-income households who have difficulty securing and maintaining rental accommodation.
	Eligibility for Rent Assistance requires eligibility for income support or more than
	the base rate of Family Tax Benefit Part A, and rental costs that exceed a minimum level.
	 (\textcite{DHS2017-Rent-assistance}).}
Maximum assistance should be indexed to move in line with market rents.
Since Rent Assistance is based on recipients' rent levels, the payment can be well-targeted to need, and the support can move with them as they move homes.

The costs would be comparatively modest.
For example, previous Grattan Institute work has recommended a targeted \$500-a-year boost to Rent Assistance for Age Pensioners as the most efficient way to alleviate financial stress among low-paid retirees, at a cost of \$250~million a year.%
	\footnote{\textcite{Daley-etal-2016-theConvo-Govt-shouldnt-use-super-for-low-income-savers}, updated to 2017-18.}
Low-income working-age households on welfare are even more stressed than low-income pensioners: boosting their Rent Assistance by \$500
would cost a further \$450~million a year.%
	\footnote{Grattan analysis of \textcites{SocialServices2015DepartmentSocialServices}{SocialServices2016DepartmentSocialServicesa}.}
	
\subsection{Management of existing social housing}\label{subsec:Management-existing-social-housing}

In the meantime, there are substantial opportunities to manage the existing social housing stock better: the stock is not well allocated to those that most need it;%
    \footcites{DHS_kpmg_social_housing_2012}{Potter-2017-Affordablehousing}
it is often not well-suited to their needs;%
    \footnote{Tenants have little choice over the home they are offered and the type of housing available can be incompatible with a recipient’s need.
    The public housing stock is dominated by three bedroom houses, yet most recipients are singles or couples without children.}
quality is often poor; and workforce participation is discouraged by using queues to ration housing assistance%
    \footnote{To remain eligible for public housing, the incomes of prospective tenants must stay low while they are on the waiting list. (\textcites[][595]{HenryTaxReview2010-Part2-Detailed-analysis}{Potter-2017-Affordablehousing}.)}
and by setting rents based on incomes.%
    \footnote{Tenants with the same income pay the same rent regardless of the size, location, condition or general amenity of the house they occupy.}
    

    










